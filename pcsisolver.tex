\section{P-CSI Solver} \label{se:psi}
%----------------------------------------------------------------------------
To address the scalability bottleneck in POP, a barotropic solver that
requires as few global reductions as possible is desired.
%Originally less efficient methods, such as Chebyshev iteration, were reconsidered in POP. 
%Chebyshev iteration was revisited by Gutknecht \cite{gutknecht2002chebyshev} in 2002, and was identified as being suitable for massively parallel computers with high communications costs. 
In \cite{hu2013scalable},  Hu et al. developed an appropriate solver based
on Stiefel's Chebyshev iteration method (CSI) and do a preliminary
evaluation at modest core counts in a standalone version of POP.
Here, we further improve the CSI by developing an effective preconditioner and implement the optimized P-CSI solver in the POP within the CESM framework. 
%P-CSI provides interfaces for different kinds of preconditioner. 
% As early as 1985, Saad et al.\cite{saad1985solving} implemented a generalization of P-CSI on a linear array of processors and claimed that this generalization is more favorable than conjugate gradient method in some cases when the eigenvalues are known.

\subsection{Algorithm and Evaluation} \label{se:psialg}
Like CSI, P-CSI does not require inner product operations and thus eliminates the bottleneck of global reduction as in PCG and ChronGear.
Instead of requiring prior knowledge about the spectrum of the coefficient matrix $A$, P-CSI needs the spectrum of the preconditioned matrix $M^{-1}A$.
For real symmetric and positive definite matrices with preconditioners,
such as the coefficient matrix $A$ and its corresponding diagonal preconditioner $M = \Lambda(A)$ in POP,
only approximations of the largest and smallest eigenvalues $\lambda_{max}$ and $\lambda_{min}$ of $M^{-1}A$ are needed to ensure the convergence of P-CSI. 
These two extreme eigenvalues can be estimated efficiently by the Lanczos method. 
The pseudo code of the new P-CSI algorithm designed for POP is shown in Algorithm 2.

\begin{algorithm}[!t]
\caption{Preconditioned Classical Stiefel Iteration solver}
\label{alg:ppsi}
\begin{algorithmic}[1]
\REQUIRE Coefficient matrix $\textbf{B}$, preconditioner $\textbf{M}$, initial guess  $\textbf{x}_0$ and right hand side vector $\textbf{b}$ associated with grid block $B_{i,j}$; Estimated eigenvalue boundary $[\nu,\mu]$;  \\
 // \qquad    \textit{do in parallel with all processes}
\STATE $\alpha =\frac{2}{\mu -\nu}$, $ \beta = \frac{\mu +\nu}{\mu -\nu}$, $\gamma = \frac{\beta}{\alpha}$, $\omega_0 =\frac{ 2}{\gamma}$;\quad $k = 0$;
\STATE $\textbf{r}_0 = \textbf{b}-\textbf{B}\textbf{x}_0$; $\textbf{x}_1 =\textbf{x}_0 -\gamma^{-1}\textbf{M}^{-1}\textbf{r}_0$; $\textbf{r}_1 =\textbf{b} -\textbf{B}\textbf{x}_1$; 
\WHILE{$k \leq k_{max}$ }
\STATE $k=k+1$;
\STATE $\omega_k = 1/(\gamma - \frac{1}{4\alpha^2}\omega_{k-1})$; \COMMENT{the iterated function}
\STATE $\textbf{r}'_{k-1} =\textbf{M}^{-1}\textbf{r}_{k-1}$; \COMMENT{preconditioning} \
\STATE $\Delta \textbf{x}_{k} =\omega_k\textbf{r}'_{k-1}+(\gamma \omega_k-1)\Delta \textbf{x}_{k-1}$; 
\STATE $\textbf{x}_{k} =\textbf{x}_{k-1}+\Delta \textbf{x}_{k-1}$; 
\STATE $\textbf{r}_{k} =b- \textbf{B}\textbf{x}_{k}$; \COMMENT{matrix--vector multiplication}
\STATE $update\_halo(\textbf{r}_k)$; \COMMENT{boundary communication}
\IF { $k \%  n_{c} == 0$ }
\STATE check convergence;
\ENDIF
\ENDWHILE
\end{algorithmic}
\end{algorithm}

The P-CSI algorithm and its properties are similar to that of the CSI
algorithm detailed in \cite{hu2013scalable}, with the exception of an
additional preconditioner, which is described in detail in the next
section. Thus, the total computation time of each iteration in the diagonal preconditioned P-CSI solver is $T_c =\frac{12\mathcal{N}^2}{p}\theta+\mathcal{T}_p =\frac{13\mathcal{N}^2}{p}\theta$
and the total execution time for each P-CSI solver step is
\begin{eqnarray}
\label{t_psi}
\mathcal{T}_{pcsi} = \mathcal{K}_{pcsi}(\mathcal{T}_c + \mathcal{T}_b ) \nonumber \\
= \mathcal{K}_{pcsi}[13\frac{\mathcal{N}^2}{p}\theta+ 4\alpha + \frac{8\mathcal{N}}{ \sqrt{p}}\beta]
\end{eqnarray}
where $K_{pcsi}$ is the number of iterations in one P-CSI solver step.  

Note that the overall convergence rate of P-CSI will be slower than
that of the PCG or ChronGear method and, 
as a result, PCSI requires a larger number of iterations ($K_{pcsi} > K_{cg}$). 
%under the same convergence tolerances.  
This will translate into a higher execution
time for P-CSI than ChronGear at smaller core counts when global
reductions are not an issue.  However, for high-resolution grids when
many cores are required, P-CSI will be notably faster than ChronGear
per iteration (see Equations (\ref{t_pcg}) and (\ref{t_psi})), which
will result a reduction in time to convergence.

\subsection{Eigenvalue Estimation for the P-CSI}
The eigenvalue estimation method for CSI \cite{hu2013scalable} is amended for P-CSI to calculate the eigenvalues of the preconditioned matrix $M^{-1}A$.
A series of tridiagonal matrices $T_m (m=1,2,...)$ whose largest ($\lambda_{max}$) and smallest ($\lambda_{min}$) eigenvalues converge to those of $M^{-1}A$
can be constructed using the Lanczos method \cite{Paige1980235} since the actual values of $\lambda_{min}$ and $\lambda_{max}$ are difficult to obtain. 
%For reference, Lanczos-based eigenvalues estimation procedure in given in Algorithm \ref{alg:lanczos_pre}. 
A more detailed discussion as related to CSI can be found in
\cite{hu2013scalable}.
%In step \ref{lanczos_converge}, $\epsilon$ is a user-defined tolerance.
In experiments, we found that  setting the Lanczos convergence tolerance $\epsilon$ to $0.15$ works efficiently in both 1\degree and 0.1 \degree cases with different preconditioners.
%$T_m$ contains $\alpha_i (i=1,2,...,m)$ as its diagonal entries and $\beta_i (i=1,2,...,m-1)$ as off-diagonal entries listed below
%\[ T_{m} = tridiag\left(\begin{array}{ccccccc}
%&\beta_1 && \bullet & &\beta_{m-1}&    \\
%\alpha_1 & &\alpha_2 && \bullet &&\alpha_{m}\\
%&\beta_1 && \bullet & & \beta_{m-1}&
%\end{array} \right)\]
Fig.\ref{fig:iter} indicates that only a small number of Lanczos steps
are necessary to generate
eigenvalue estimates of $M^{-1}A$ that result in near-optimal P-CSI
convergence. 


%%IF WE NEED MORE SPACE WE CAN LEAVE OUT ALGORITHM 3
%convergence speed of P-CSI reaches its theoretical optimum when $\nu = \lambda_{min}$ and $\mu =\lambda_{max}$. 
%Accurate values of $\lambda_{min}$ and $\lambda_{max}$ are difficult to obtain. In addition, any transformation of the coefficient matrix $A$ is ill-advised because $A$ was distributed to processes.
%To utilize the parallelism of POP, we employ Lanczos method  to construct
%In practice, we find that this theoretical optimum has a iteration number close to the one of PCG.


%\begin{algorithm}[!ht]
%\caption{Lanczos-based Eigenvalue Estimation for Preconditioned Matrix}
%\label{alg:lanczos_pre}
%\begin{algorithmic}[1]
%\REQUIRE Coefficient matrix $\textbf{B}$, preconditioner $\textbf{M}$, and random vector $\textbf{r}_0$ associated with grid block $B_{i,j}$; \\
% //\qquad    \textit{do in parallel with all processes}
%\STATE $\textbf{s}_0=\textbf{M}^{-1}\textbf{r}_0$;\quad $\textbf{q}_1 = \textbf{r}_0/({\textbf{r}_0^T\textbf{s}_0})$;\quad $\textbf{q}_0=\textbf{0}$;
%\STATE $T_0=\emptyset$;\quad $\beta_0 =0$;\quad  $\mu_0 =0$;\quad $j=1$;
%\WHILE{$j<k_{max}$}
%\STATE $\textbf{p}_j = \textbf{M}^{-1}\textbf{q}_j$; \quad $\textbf{r}_j=\textbf{B}\textbf{p}_j-\beta_{j-1}\textbf{q}_{j-1}$;
%\STATE $update\_halo(\textbf{r}_j)$;
%\STATE $\tilde{\alpha}_j =\textbf{p}_j^T\textbf{r}_j$; \quad $\alpha_j=global\_sum(\tilde{\alpha}_j)$;
%\STATE $\textbf{r}_j=\textbf{r}_j-\alpha_{j}\textbf{q}_{j}$; \quad $\textbf{s}_j = \textbf{M}^{-1}\textbf{r}_j$; 
%\STATE $\tilde{\beta}_j = \textbf{r}_j^T\textbf{s}_j$; \quad $\beta_j=sqrt(global\_sum(\tilde{\beta}_j))$;
%\STATE \textbf{if} $\beta_j == 0$ \textbf{then} \textbf{return}
%\STATE $\mu_j = max(\mu_{j-1},\alpha_j+\beta_j+\beta_{j-1})$; \label{lan_gersh}\\
%\STATE $T_j=tri\_diag(T_{j-1},\alpha_j,\beta_j)$; \COMMENT{Tridiagonal}\label{lan_tm}
%\STATE $\nu_j = eigs(T_j,'smallest')$ ; \label{lan_nu}
%\STATE \textbf{if} $|\frac{\mu_j}{\mu_{j-1}} -1 |< \epsilon\quad\textbf{and}\quad|1- \frac{\nu_j}{\nu_{j-1}}|< \epsilon$ \textbf{then} \textbf{return}; \label{lanczos_converge}
%\STATE $\textbf{q}_{j+1}= \textbf{r}_j/\beta_j$;\quad $j=j+1$;
%\ENDWHILE
%\end{algorithmic}
%\end{algorithm}

%This algorithm assume that $\textbf{A}$ is positive defined symmetrical matrix.  However, it is simple to adjust the algorithm for negative defined matrix as in our case. Just negating both $\textbf{A}$ and its preconditioner. 

\begin {figure}[!t]
\centering
\includegraphics[width=1.0\linewidth]{solver_iteration}
\caption []{Relationship of P-CSI iterations and Lanczos steps in 1\degree POP \label {fig:iter}}
\end {figure}

