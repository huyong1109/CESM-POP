\section{Experiments} \label{se:exp}
%----------------------------------------------------------------------------
%\subsection{Experimental Platform} \label{se:plat}

The performance of our new barotropic solver is tested in CESM1.2.0 on the Yellowstone and Edison Supercomputers. Yellowstone Supercomputer is located at NCAR-Wyoming Supercomputing Center (NWSC) in Cheyenne, Wyoming,USA. It contains 72,576 Intel Sandy Bridge processors, which are connected by a 13.6 GBps InfiniBand network. Processors are 2.6-GHz Intel Xeon E5-2670 with Advanced Vector Extensions (AVX). 

To focus on the performance of POP, the G\_NORMAL\_YEAR component set which couples ocean and ice with COREv2 normal year forcing is used.  
Two grid resolutions for POP are examined:
\begin{itemize} 
\item 1\degree resolution horizontal grid ($320\times 384$)
\item 0.1\degree resolution horizontal grid ($3600\times 2400$)
\end{itemize}
The suggested space-filling curve method for distributing blocks across processors is used in 0.1\degree case. 
%\begin{center}
%\begin{tabular}{|c|c|c|c|c|c|c|}
%\hline
%Platform & Processor &Speed    & Cores & Memory & Cache & Network \\
%\hline
%TS100 & Intel Xeon  & 2.93GHz & 12   &    24/48G   & 12M & InfiniBand QDR \\
%\hline
%SW  	  & SW1600    & 1.1GHz   & 16   & 16G       & N/A  & InfiniBand QDR\\
%\hline
%Yellowstone  & Intel Xeon    & 2.6GHz   & 16   & 32G       & 20M  & InfiniBand FDR \\
%\hline
%\end{tabular}
%\end{center}


\subsection{Low Resolution Simulations}
\begin {figure}[!t]
\centering
\includegraphics[width=1.0\linewidth]{1deg_solverruntime}
\caption []{Execution time of barotropic solvers in the 1\degree POP for one day.\label {fig:runtime1}}
\end {figure}

The execution time of the barotropic solvers in 1\degree POP on a series of cores on Yellowstone is shown in Fig.\ref{fig:runtime1}. 
P-CSI out-performs ChronGear on different processor cores and reduces the solver execution time from 0.60s to 0.44s per simulation day, that is a 1.4X speedup. 
EVP preconditioning provides speedup for both ChronGear and P-CSI. In the case of 384 cores, EVP preconditioner achieves a speed of 0.33s per simulation day which is 1.8X of the original ChronGear solver. 

\begin{table}[!t]
\centering
\caption{Percentage of improvement of the execution time in 1\degree POP. \label{tab:improve_1}}
\begin{tabular}{|l||r|r|r|r|}
\hline
Number of cores & 48  & 96  & 192& 384 \\\hline
\hline
ChronGear+EVP &0.1\% &0.9\%	&5.8\%&7.9\% \\\hline
P-CSI+Diagonal  & 1.5\% &2.4\%&7.9\%  &7.9\% \\\hline
P-CSI+EVP	      &0.3\%& 2.3\%	&10.0\%  &13.0\% \\\hline
\end{tabular}
\end{table}
The improvement of the barotropic solver contributes to accelerating the whole model. 
Detailed improvement of the whole model compared with the original diagonal preconditioned ChronGear solver would be found in Table \ref{tab:improve_1}. The combination of P-CSI and EVP preconditioner improves the whole model by 13.0\% on 384 processor cores. 

\subsection{High Resolution Simulations}
\begin {figure*}[t!]
\begin{center}
\includegraphics[width =0.45\linewidth]{01deg_solverruntime}
\hspace{10pt}
\includegraphics[width =0.45\linewidth]{01deg_speedup_ys}
\end{center}
\caption []{Execution time of barotropic solvers in the 0.1\degree POP for one day on Yellowstone (left). 
The relative speedup of the whole POP model referred to the original configuration with a diagonal preconditioned ChronGear solver on 470 cores on Yellowstone (right).\label {fig:runtime01}}
\end {figure*}

To test the scalability of the new barotropic solver, we first run 0.1\degree POP on both Yellowstone.
As shown in Fig.\ref{fig:runtime01}, ChronGear stops scaling after 2,700 cores, while the execution time of P-CSI keeps decreasing until 10,800 cores are used. 
The implementation of P-CSI in the 0.1\degree resolution POP accelerates the barotropic mode by 5.1 times, from 21.8s to 4.3s per simulation day, on 16,875 cores.
EVP preconditioning provides a further acceleration to both ChronGear and P-CSI. 
The combination of ChronGear and EVP preconditioning accelerates the barotropic solver by 1.5 times,
while the combination of P-CSI and EVP preconditioning provides a speedup of 7.1 times.
\begin {figure}
\centering
\vspace{-10pt}
\includegraphics[width=1.0\linewidth]{POPStepComp_pcsi.eps}
\caption[] {Percentage of execution time in 0.1\degree POP using the P-CSI solver.\label{fig:StepComp_pcsi}}
\vspace{-10pt}
\end{figure}

As talked about in section \ref{se:baro}, execution time of the original barotropic solver takes an increasing percentage of the whole model. 
However, Fig.\ref{fig:StepComp_pcsi} demonstrates that the barotropic mode would remains a small part of the whole model when the scalable EVP preconditioned P-CSI solver is implemented.  
In contrast to the original ChronGear solver, the new barotropic solver takes up only 12.4\% of the total execution time instead of 49.3\% when 16,875 cores are used. 

Consequently, improvement in barotropic mode will benefit the speed of the whole POP model, especially when large numbers of processor cores are used. 
Fig.\ref{fig:runtime01} (left) shows that diagonal preconditioned and EVP preconditioned P-CSI solver accelerate the whole POP model by 1.66 and 1.74 times respectively. 


%Detailed improvement of the 0.1 \degree POP  would be found in Table \ref{tab:improve_01}. 
%The combination of P-CSI and EVP preconditioner improves the whole model by 25.7\% on 9128 processor cores. 
%EVP preconditioning contributes 7.8\% improvement to the original ChronGear solver, as well as 3.4\% extra improvement to P-CSI solver. 
%
\begin{figure*}[!t]
\begin{center}
	\includegraphics[width=1.0\linewidth]{01solvercomp.eps}
\end{center}
\caption[] {Time components of barotropic solvers in 0.1\degree POP. Four kinds of solvers are ChronGear or P-CSI with a diagonal preconditioner or an EVP preconditioner. (a) Execution time of global reduction, (b) Execution time of boundary communication, (c) Execution time of computing. }
\label{fig:component}
\end{figure*}

%\begin{table*}
%\centering
%\caption {Percentage of improvement of the total execution time in 0.1\degree POP. \label{tab:improve_01}}
%\begin{tabular}{|l||r|r|r|r|r|r|r|r|}
%%\toprule
%\hline
%Number of cores & 512  & 1224   & 2634 & 3476 & 4028 & 6124 & 9128 & 12330\\\hline
%\hline
%ChronGear+Evp &-0.3\% &0.2\%&3.4\%  & 6.2\%& 8.3\% & 10.7\%& 12.4\% & 14.0\%\\\hline
%P-CSI+Diagonal  & 1.3\% & 2.1\%	&7.5\%  &12.1\% &15.6\% &23.2\% &22.3\% &30.5\%\\\hline
%P-CSI+Evp	      &-0.2\% & 1.1\%&6.5\%  &12.2\% &16.5\% &26.0\% &25.7\% &35.0\%\\
%\hline
%%\bottomrule
%\end{tabular}
%\end{table*}
\begin {figure*}[t!]
\begin{center}
\includegraphics[width =0.45\linewidth]{01deg_solverruntime_edison}
\hspace{10pt}
\includegraphics[width =0.45\linewidth]{01deg_speedup_edison}
\end{center}
\caption []{Execution time of barotropic solvers in the 0.1\degree POP for one day on Edison (left). 
The relative speedup of the whole POP model referred to the original configuration with a diagonal preconditioned ChronGear solver on 470 cores on Edison (right).\label {fig:runtime01_edison}}
\end {figure*}

Simulation rate (simulated years per wall-clock day) is a popular criterion for model performance. 
Here we use the ideal simulation rate (only the execution time of STEP function in POP counts) in order to get rid of the effect from initialization, preprocessing  and disk I/O. 
Table \ref{tab:improve_01} shows that to satisfy a rate of 5 simulated years per wall-clock day, which is the minimum necessary simulation rate to run long term climate simulations, 
ChronGear requires more than 7,500 cores while P-CSI requires only 4,220 cores. 
The EVP preconditioned P-CSI solver improves the ideal simulation rate of POP by 1.74 times on 16,875 cores, from 5.34 to 9.24 simulated years per wall-clock day. 

To understand the source of improvement, more detail about the barotropic solver is provided in Fig.\ref{fig:component}.
Fig.\ref{fig:component} tells that P-CSI outperforms ChronGear mainly because of less global reduction. 
Even though EVP preconditioning almost doubles the execution time of computation, it helps to reduce both boundary and global communication. 
%Exception happens when small number of cores are used, due to load imbalance. 
It is interesting that when less than 4,000 cores are used, the execution time of global reduction and boundary communication does not decrease after replacing the diagonal preconditioner with EVP preconditioner for P-CSI. 
This is mainly due to the load imbalance of POP. 
The less the number of cores are used, the bigger the block size on each process is. 
Because of the existence of land, bigger block size leads to more imbalance between blocks.   
This explains why the execution time of global reduction decreases when less than 1,224 cores are used, which conflicts with the theoretical result in equation \ref{t_pcg} and \ref{t_psi}. 
EVP preconditioning enlarges this load imbalance because it almost doubles the execution time of computation.
Thus, even though the iteration number is reduced to a half in the case of EVP preconditioned P-CSI, each communication takes longer to complete 

\subsection{Platform Edison}
The 0.1 \degree POP is run on Edison to duplicate the scaling performance. 
Edison is the newest supercomputer of the National Energy Research Scientific Computing Center (NERSC), 
which has a peak performance of 2.57 PFLOPS, 133,824 2.4 GHz Intel Ivy Bridge processor cores connected by an 8GBps Cray Aries high-speed interconnect with Dragonfly topology. 

As shown in Fig.\ref{fig:runtime01_edison}, simulations on Edison with four solver configurations have similar performance as on Yellowstone. ChronGear stops scaling after 2,700 cores, while P-CSI scales well until 4,220 cores are used. 
It is worth mentioning that global communication is more unstable on Edison than on Yellowstone. As a result, the execution time of ChronGear with either a diagonal preconditioner or an EVP preconditioner varies a lot from run to run. Here we use the average of the best three results to represent the execution time. 
The implementation of P-CSI in the 0.1\degree resolution POP accelerates the barotropic mode by 4 times, from 22.2s to 5.5s per simulation day, on 16,875 cores.
EVP preconditioning provides a further acceleration to both ChronGear and P-CSI. 
The combination of ChronGear and EVP preconditioning accelerates the barotropic solver by 2.5 times, 
while the combination of P-CSI and EVP preconditioning provides a speedup of 6.5 times. 
 
\begin{table}
\vspace{-10pt}
\centering
\caption {Ideal simulation rate of 0.1\degree POP \label{tab:improve_01}}
\begin{tabular}{|l||r|r|r|r|r|r|r|}
%\toprule
\hline
Solver & 470  & 1200   & 2700 & 4220 & 7500 & 10800 & 16875\\\hline
\hline
ChronGear     &0.71 &1.68&3.38  &4.62 &6.02 &6.07 &5.34\\\hline
ChronGear+Evp &0.70 &1.69&3.44  &4.88 &6.62 &6.89 &6.46\\\hline
P-CSI+Diag    &0.72 &1.72&3.51  &5.00 &7.04 &8.29 &8.85\\\hline
P-CSI+Evp     &0.70 &1.69&3.49  &5.01 &7.23 &8.55 &9.27\\
\hline
%\bottomrule
\end{tabular}
\end{table}
