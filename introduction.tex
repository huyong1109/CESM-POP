\section{Introduction} \label{se:int}
% no \IEEEPARstart
High-resolution global climate models have become increasing important
in recent years as a means for
understanding past climate variability and projecting future climate
change.  The Community Earth System Model (CESM),
whose development is centered at the National Center for Atmospheric
Research (NCAR), is one of the most widely used global climate models,
and its climate projections are a key component in the Intergovernmental
Panel on Climate Change (IPCC) Fifth Assessment Report (AR5)
\cite{stocker2013ipcc}.

CESM is a fully-coupled climate model system, including
atmosphere, ocean, sea-ice and land components.  In contrast with the
atmosphere and land, the ocean has a broader spatial scale, ranging
from tens of kilometers of the ocean basins into the sub-mesoscale
eddies, and a broader temporal scale, ranging from the order of days
for gravity waves to centuries for slow planetary waves.  As a result,
ocean models have to iterate for a longer time at a finer resolution,
which makes them the most computationally intensive part of climate
models \cite{Worley:2011:PCE:2063384.2063457,dennis2012computational}.
Many recent studies demonstrate that high-resolution ocean models are
required in order to produce more realistic and accurate predictions
\cite{bryan2010frontal,mcclean2011prototype,graham2014importance}.
 

Moreover, climate model simulations are often performed for decades or
even centuries, however, these long-term simulation are too
computationally expensive to run at high-resolution.  For example,
most CESM simulations in IPCC AR5 are carried out with a nominal
1\degree ocean model and a 1\degree to 2\degree atmosphere model.
Recently, the increasing computational power of supercomputers and
high-resolution satellite observations have inspired much research
that focuses on adapting high-resolution climate models for massive
parallelism.

As noted, in most production simulations, the least scalable 
component of CESM is the ocean model Parallel Ocean Model (POP)
\cite{dennis2012computational}.  POP solves the three-dimensional
primitive equations with hydrostatic and Boussinesq approximations
\cite{smith2010parallel}.  The time integration is divided into two
parts: the baroclinic mode and the barotropic mode.  The implicit
free-surface method is commonly used in the barotropic mode ocean
models because it allows a large time step to efficiently compute the
fast gravity mode.  However, this method requires the solution of a
large elliptic system of equations, which can potentially cause poor
scalability for the barotropic mode in the parallelization.  In fact,
much attention has focused on the poor scaling performance of the
barotropic mode in POP,  which is dominated by
the communication overhead \cite{Worley:2011:PCE:2063384.2063457}, and
it is recognized that
optimization of the barotropic mode of POP will benefit the whole
climate model  \cite{dennis2012computational} .


The Chronopoulos-Gear (ChronGear) method \cite{dAzevedo1999lapack},
which is a modified Preconditioned Conjugate Gradient method (PCG),  is
the default linear solver for the POP barotropic mode.  For
high-resolution simulations, the required global reduction results in
poor scalability. To improve the scaling of POP, and, therefore CESM, 
we focus on optimizing the barotropic solver.  In particular, we
eliminate the global reductions, which is a major impediment to 
scalability.  We make the following contributions:


%MOVE TO RELATED WORK
%There are currently different alternatives to mitigate the poor behavior of the PCG type of solver in the massive parallelization.
%Some approaches attempt to overlap the communication with computation time\cite{beare1997optimisation}.
%Some use the land elimination and load-balance strategies \cite{dennis2007inverse, dennis2008scaling}
%to reduce the number of processes and the associated overhead of global reduction.
%Although these approaches may improve performance, they do not eliminate the major bottleneck of the global reduction.



\begin{itemize}
\item We improve the convergence of the Classical Stiefel Iteration 
(CSI) solver proposed in \cite{hu2013scalable} with a preconditioning interface.
The resulting preconditioned CSI (P-CSI) solver significantly reduces
 the number of CSI iterations at each time step.
\item We develop an effective parallel preconditioner based on the
Error Propagation method (EVP) \cite{roache1995elliptic}.
\item
We implement the P-CSI solver (and EVP preconditioner)  in CESM1.2 and evaluate its performance.
\item
We apply an ensemble-based statistical method to evaluate the impact
of the new P-CSI solver and ensure that a consistent climate is produced.
\end{itemize}


 The remainder of this paper is organized as follows.

 Section \ref{se:baro} briefly introduces the barotropic solver of POP
 and evaluates the scalability of the iterative methods
 theoretically. %{\bf what a communication model?}
 Section \ref{se:psi} and \ref{se:evp} detail the design of P-CSI in
 POP and the EVP preconditioner.  Section \ref{se:exp} compares the
 scalability of the ChronGear and P-CSI solvers.  Section \ref{se:ver}
 verifies the new P-CSI solver using the ensemble based statistical
 method.  Finally, related work is described in section \ref{se:rel}
 and the conclusion is presented in section \ref{se:conc}.

%{\textbf better to move related work in the next section.}

% You must have at least 2 lines in the paragraph with the drop letter
% (should never be an issue)
