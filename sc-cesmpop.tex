%
% ================= IF YOU HAVE QUESTIONS =======================
% Questions regarding the SIGS styles, SIGS policies and
% procedures, Conferences etc. should be sent to
% Adrienne Griscti (griscti@acm.org)
%
% Technical questions _only_ to
% Gerald Murray (murray@hq.acm.org)
% ===============================================================
%
% For tracking purposes - this is V2.0 - May 2012

\documentclass{sig-alternate}
\usepackage{gensymb}  %format of algorithm
\usepackage{algorithm}  %format of algorithm
\usepackage{algorithmic}  %format of algorithm
\renewcommand{\algorithmiccomment}[1]{ \hfill {/* #1 */} }
\usepackage{graphicx}
\usepackage{epstopdf}
\DeclareGraphicsExtensions{.eps,.mps,.pdf,.jpg,.PNG}
\DeclareGraphicsRule{*}{pdf}{*}{}
\graphicspath{{./fig/}}

\newcommand{\superscript}[1]{\ensuremath{^{\textrm{#1}}}}
\def\sharedaffiliation{\end{tabular}\newline\begin{tabular}{c}}
\def\thua{\superscript{\dag}}
\def\thub{\superscript{\ddag}}
\def\ncara{\superscript{*}}
\def\ncarb{\superscript{\P}}

\begin{document}
%
% --- Author Metadata here ---
\conferenceinfo{SC'15}{November 15-20, 2015, Austin, Texas, USA.}
%\CopyrightYear{2007} % Allows default copyright year (20XX) to be over-ridden - IF NEED BE.
%\crdata{0-12345-67-8/90/01}  % Allows default copyright data (0-89791-88-6/97/05) to be over-ridden - IF NEED BE.
% --- End of Author Metadata ---

%\title{Alternate {\ttlit ACM} SIG Proceedings Paper in LaTeX
%Format\titlenote{(Produces the permission block, and
%copyright information). For use with
%SIG-ALTERNATE.CLS. Supported by ACM.}}
%\subtitle{[Extended Abstract]
%\titlenote{A full version of this paper is available as
%\textit{Author's Guide to Preparing ACM SIG Proceedings Using
%\LaTeX$2_\epsilon$\ and BibTeX} at
%\texttt{www.acm.org/eaddress.htm}}}

\title{Improving the Scalability of the Ocean Barotropic Solver \\in the Community Earth System Model}
%
% You need the command \numberofauthors to handle the 'placement
% and alignment' of the authors beneath the title.
%
% For aesthetic reasons, we recommend 'three authors at a time'
% i.e. three 'name/affiliation blocks' be placed beneath the title.
%
% NOTE: You are NOT restricted in how many 'rows' of
% "name/affiliations" may appear. We just ask that you restrict
% the number of 'columns' to three.
%
% Because of the available 'opening page real-estate'
% we ask you to refrain from putting more than six authors
% (two rows with three columns) beneath the article title.
% More than six makes the first-page appear very cluttered indeed.
%
% Use the \alignauthor commands to handle the names
% and affiliations for an 'aesthetic maximum' of six authors.
% Add names, affiliations, addresses for
% the seventh etc. author(s) as the argument for the
% \additionalauthors command.
% These 'additional authors' will be output/set for you
% without further effort on your part as the last section in
% the body of your article BEFORE References or any Appendices.

%\numberofauthors{6} 
%
%\author{
%\alignauthor 
%Yong Hu
%	\email{huyong11@mails.tsinghua.edu.cn}
%\alignauthor 
%Yu-heng Tseng
%       \email{ytseng@ucar.edu}
%\alignauthor 
%Allison Baker
%       \email{abaker@ucar.edu}
%\and  % use '\and' if you need 'another row' of author names
%\alignauthor 
%Xiaomeng Huang
%       \email{hxm@tsinghua.edu.cn}
%\alignauthor 
%Frank Bryan
%       \email{bryan@ucar.edu}
%\alignauthor 
%Guangwen Yang 
%       \email{ygw@tsinghua.edu.cn}
%}

\numberofauthors{1} 
\author{
\alignauthor 
Yong Hu\thua\thub,Yu-heng Tseng\ncara,Allison H. Baker\ncarb, \\Xiaomeng Huang\thub, Frank O. Bryan\ncara, Guangwen Yang\thua\thub 
\sharedaffiliation
  \begin{tabular}{ccc}
    \affaddr{{\thua}Department of Computer Science \& Technology {\ }} & & \affaddr{{\ncara}Climate and Global Dynamics Division {\ }} \\
    \affaddr{{\thub} Center for Earth System Science{\ } } & & \affaddr{{\ncarb}Computational and Information Systems Laboratory{\ }} \\    
    \affaddr{Tsinghua University,Beijing,China} & & \affaddr{National Center for Atmospheric Research} \\
    \email{huyong11@mails.tsinghua.edu.cn} & & \affaddr{Boulder,CO. USA} \\
    \email{\{hxm,ygw\}@tsinghua.edu.cn}  & & \email{\{ytseng,abaker,bryan\}@ucar.edu} \\
  \end{tabular}
}

 

%\sharedaffiliation
%  \begin{tabular}{ccc}
%    \affaddr{{\thu}Ministry of Education Key Laboratory for Earth System Modeling{\ }} & & \affaddr{{\ncar}Climate and Global Dynamics Division, Computational and Information Systems Laboratory{\ }} \\
%    \affaddr{Center for Earth System Science, Tsinghua University, Beijing, 100084, China}            & & \affaddr{National Center for Atmospheric Research} \\
%    \affaddr{Tsinghua National Laboratory for Information Science and Technology (TNList)}            & & \affaddr{1850 Table Mesa Dr., Boulder, CO80305, USA} \\
%    \affaddr{United States}                  & & \affaddr{United States} \\
%  \end{tabular}
%

%\numberofauthors{6} %  in this sample file, there are a *total*
%\author{
%%
%% 1st. author
%\alignauthor
%Ben Trovato
%       \email{trovato@corporation.com}
%% 2nd. author
%\alignauthor
%G.K.M. Tobin
%       \email{webmaster@marysville-ohio.com}
%% 3rd. author
%\alignauthor Lars Th
%       \email{larst@affiliation.org}
%\and  % use '\and' if you need 'another row' of author names
%% 4th. author
%\alignauthor Lawrence P. Leipuner
%       \email{lleipuner@researchlabs.org}
%% 5th. author
%\alignauthor Sean Fogarty
%       \email{fogartys@amesres.org}
%% 6th. author
%\alignauthor Charles Palmer
%       \email{cpalmer@prl.com}
%}
% There's nothing stopping you putting the seventh, eighth, etc.
% author on the opening page (as the 'third row') but we ask,
% for aesthetic reasons that you place these 'additional authors'
% in the \additional authors block, viz.
%\additionalauthors{Additional authors: John Smith (The Th{\o}rv{\"a}ld Group,
%email: {\texttt{jsmith@affiliation.org}}) and Julius P.~Kumquat
%(The Kumquat Consortium, email: {\texttt{jpkumquat@consortium.net}}).}
\date{30 March 2015}
% Just remember to make sure that the TOTAL number of authors
% is the number that will appear on the first page PLUS the
% number that will appear in the \additionalauthors section.

\maketitle
\begin{abstract}
The increasing need for high-resolution climate simulations warrants the investigation of a new preconditioned linear solver for the barotropic mode in the Parallel Ocean Model (POP).  
POP is the ocean component of the Community Earth System Model (CESM) framework and is computationally expensive for high-resolution grids (e.g., 0.1\degree) on many cores.  
Similar to other free-surface models, POP solves the fast barotropic dynamics with the implicit free-surface method.  
This method requires the solution of an elliptic system of equations, and POP solves this elliptic system with a modified Preconditioned Conjugate Gradient (PCG) method.
While effective at moderate core counts and resolutions, PCG methods scale poorly on large parallel systems due to the global reductions required by the inner product calculations.  
This poor scaling is problematic for high-resolution CESM simulations.  
In this work, we implement a more scalable Preconditioned Stiefel Iteration method (P-CSI) into the barotropic mode of POP within CESM, and develop an effective Error Vector Propagation preconditioner to improve the convergence rate.  
Our experiments show that P-CSI accelerates the high-resolution (0.1\degree)  barotropic solver by 7.1 times on 16,875 cores, from 21.84s to 3.06s per simulation day.
This improves the core POP simulation (without I/O and initialization) by 1.74 times, from 5.3 to 9.3 simulated years per wall-clock day.  
Further, to ensure that the use of an alternative solver in POP does not negatively impact the ocean climate, we use an ensemble-based statistical method to evaluate the impact of the new solver.

\end{abstract}
% A category with the (minimum) three required fields
\category{J.2}{Physical Sciences and Engineering}{Earth and atmospheric sciences}
%A category including the fourth, optional field follows...
\category{D.1.3}{Programming Techniques}{Parallel Programming}
\terms{Algorithms,Performance}
\keywords{Massive Parallelism, Preconditioned Conjugate Gradient, Classical Stiefel Iteration, Parallel Ocean Program}

\section{Introduction} \label{se:int}
% no \IEEEPARstart
High-resolution global climate models have become increasing important
in recent years as a means for
understanding past climate variability and projecting future climate
change.  The Community Earth System Model (CESM),
whose development is centered at the National Center for Atmospheric
Research (NCAR), is one of the most widely used global climate models,
and its climate projections are a key component in the Intergovernmental
Panel on Climate Change (IPCC) Fifth Assessment Report (AR5)
\cite{stocker2013ipcc}.

CESM is a fully-coupled climate model system, including
atmosphere, ocean, sea-ice and land components.  In contrast with the
atmosphere and land, the ocean has a broader spatial scale, ranging
from tens of kilometers of the ocean basins into the sub-mesoscale
eddies, and a broader temporal scale, ranging from the order of days
for gravity waves to centuries for slow planetary waves.  As a result,
ocean models have to iterate for a longer time at a finer resolution,
which makes them the most computationally intensive part of climate
models \cite{Worley:2011:PCE:2063384.2063457,dennis2012computational}.
Many recent studies demonstrate that high-resolution ocean models are
required in order to produce more realistic and accurate predictions
\cite{bryan2010frontal,mcclean2011prototype,graham2014importance}.
 

Moreover, climate model simulations are often performed for decades or
even centuries, however, these long-term simulation are too
computationally expensive to run at high-resolution.  For example,
most CESM simulations in IPCC AR5 are carried out with a nominal
1\degree ocean model and a 1\degree to 2\degree atmosphere model.
Recently, the increasing computational power of supercomputers and
high-resolution satellite observations have inspired much research
that focuses on adapting high-resolution climate models for massive
parallelism.

As noted, in most production simulations, the least scalable 
component of CESM is the ocean model Parallel Ocean Model (POP)
\cite{dennis2012computational}.  POP solves the three-dimensional
primitive equations with hydrostatic and Boussinesq approximations
\cite{smith2010parallel}.  The time integration is divided into two
parts: the baroclinic mode and the barotropic mode.  The implicit
free-surface method is commonly used in the barotropic mode ocean
models because it allows a large time step to efficiently compute the
fast gravity mode.  However, this method requires the solution of a
large elliptic system of equations, which can potentially cause poor
scalability for the barotropic mode in the parallelization.  In fact,
much attention has focused on the poor scaling performance of the
barotropic mode in POP,  which is dominated by
the communication overhead \cite{Worley:2011:PCE:2063384.2063457}, and
it is recognized that
optimization of the barotropic mode of POP will benefit the whole
climate model  \cite{dennis2012computational} .


The Chronopoulos-Gear (ChronGear) method \cite{dAzevedo1999lapack},
which is a modified Preconditioned Conjugate Gradient method (PCG),  is
the default linear solver for the POP barotropic mode.  For
high-resolution simulations, the required global reduction results in
poor scalability. To improve the scaling of POP, and, therefore CESM, 
we focus on optimizing the barotropic solver.  In particular, we
eliminate the global reductions, which is a major impediment to 
scalability.  We make the following contributions:


%MOVE TO RELATED WORK
%There are currently different alternatives to mitigate the poor behavior of the PCG type of solver in the massive parallelization.
%Some approaches attempt to overlap the communication with computation time\cite{beare1997optimisation}.
%Some use the land elimination and load-balance strategies \cite{dennis2007inverse, dennis2008scaling}
%to reduce the number of processes and the associated overhead of global reduction.
%Although these approaches may improve performance, they do not eliminate the major bottleneck of the global reduction.



\begin{itemize}
\item We improve the convergence of the Classical Stiefel Iteration 
(CSI) solver proposed in \cite{hu2013scalable} with a preconditioning interface.
The resulting preconditioned CSI (P-CSI) solver significantly reduces
 the number of CSI iterations at each time step.
\item We develop an effective parallel preconditioner based on the
Error Propagation method (EVP) \cite{roache1995elliptic}.
\item
We implement the P-CSI solver (and EVP preconditioner)  in CESM1.2 and evaluate its performance.
\item
We apply an ensemble-based statistical method to evaluate the impact
of the new P-CSI solver and ensure that a consistent climate is produced.
\end{itemize}


 The remainder of this paper is organized as follows.

 Section \ref{se:baro} briefly introduces the barotropic solver of POP
 and evaluates the scalability of the iterative methods
 theoretically. %{\bf what a communication model?}
 Section \ref{se:psi} and \ref{se:evp} detail the design of P-CSI in
 POP and the EVP preconditioner.  Section \ref{se:exp} compares the
 scalability of the ChronGear and P-CSI solvers.  Section \ref{se:ver}
 verifies the new P-CSI solver using the ensemble based statistical
 method.  Finally, related work is described in section \ref{se:rel}
 and the conclusion is presented in section \ref{se:conc}.

%{\textbf better to move related work in the next section.}

% You must have at least 2 lines in the paragraph with the drop letter
% (should never be an issue)

\section{Barotropic Solver} \label{se:baro}
%----------------------------------------------------------------------------
POP solves three-dimensional primitive equations with hydrostatic and Boussinesq approximations.
The time integration separates the dynamics into baroclinic and barotropic modes.
The baroclinic mode describes the three-dimensional dynamic process while
the barotropic mode solves the vertically-integrated momentum and continuity equations \cite{smith2010parallel}.

The most time-consuming portion of the barotropic simulation is the solution of the elliptic system for the sea surface height (SSH)
due to the implicit free-surface algorithm\cite{pop05}. 

The implicit elliptic equations of SSH in POP can be expressed as follows:
\begin{equation}
\label{eq:ssh}
[\nabla \cdot H\nabla  -\phi(\tau)]\eta^{n+1} = \psi(\eta^n,\eta^{n-1},\tau)
\end{equation}
where $H$ is the depth of the ocean, $\tau$ is the time step  and $\eta^n$ is the SSH at the n-th time step, and $\psi$ represents a function of previous states.
Equation (\ref{eq:ssh}) is then discretized on a two-dimensional grid using a nine-point stencil in POP. The stencil can be reorganized into a linear system $Ax =b$.
%$A_{i,j}^0$, $A_{i,j}^n$, $A_{i,j}^e$ and $A_{i,j}^{ne}$ are symmetrical coefficients between grid point $(i,j)$ and its neighbors, determined by $H$, $\tau$ and the grid lengths. The stencil confined to grid point $(i,j)$ is
%\begin{eqnarray}
%\label{eq:sten}
%&A_{i,j}^0\eta_{i,j}+A_{i,j}^e\eta_{i+1,j}+A_{i,j}^n\eta_{i,j+1} +A_{i,j}^{ne}\eta_{i+1,j+1}\nonumber\\
%&+A_{i-1,j}^{ne}\eta_{i-1,j+1} +A_{i-1,j}^e\eta_{i-1,j}+A_{i-1,j-1}^{ne}\eta_{i-1,j-1}\nonumber\\
%&+A_{i,j-1}^n\eta_{i,j-1}+ A_{i+1,j-1}^{ne}\eta_{i,j-1}=\psi_{i,j}
%\end{eqnarray}

For massive parallelism, POP divides the global domain into blocks and distributes them to processes. 
Each process only computes the evolution procedures related to the grids in its own block, and maintains a halo region to update data with its neighbors.


For simplicity, we can assume that the global domain size is $\mathcal{N}\times \mathcal{N}$,
which can be divided into $m\times m$ blocks with size of $n\times n$ ($n=\mathcal{N}/m$). 
Let $B$ and $\tilde{x}$ be the coefficient matrix and vector associated with a given block (i.e., a sub-matrix of $A$ with a size of $n^2\times n^2$),
then the matrix-vector multiplication of $B\tilde{x}$ has $9n^2$ multiplication operations \cite{hu2013scalable}.
%----------------------------------------------------------------------------
\subsection{ChronGear Solver }
ChronGear \cite{dAzevedo1999lapack} is a modified conjugate gradient method.
%with a diagonal preconditioner $M = \Lambda(A)$.
It combines two global reductions into one, which achieves a one-third
latency reduction when solving the linear system in POP that
approximates the barotropic mode.

However, as previously mentioned, when thousands of cores are used in high-resolution (0.1\degree) POP,
the global reduction needed by the inner product in ChronGear becomes a major bottleneck.
This decrease in performance is demonstrated in
Fig.\ref{fig:StepComp} for  0.1\degree POP.  The execution time for the linear solve increases as the number of processor cores increases. 
When 512 cores are used, the execution time of the barotropic solver is only 4\% of the total execution of POP, which is much smaller than the baroclinic mode time (90\%). 
However, when a few thousand cores are used, the baroclinic execution
time appropriately decreases, while the barotropic solver time
increases and exhibits poor scalability. 
When over six thousand cores are used, the percentage of the
barotropic solver execution time reaches 32\%, and the baroclinic mode
has been reduced to 50\%.   This trend continues at higher core counts.

\begin {figure}[!t]
\vspace{-5pt}
\centering
\includegraphics[width=1.0\linewidth]{POPStepComp}
\caption[] {Percentage of execution time in 0.1\degree POP using the default ChronGear solver.\label{fig:StepComp}}
\end{figure}


\begin{algorithm}[!h]
\caption{Chronopoulos-Gear Solver}
\label{alg:pcg}
\begin{algorithmic}[1]
\REQUIRE Coefficient matrix $\textbf{B}$, preconditioner $\textbf{M}$, initial guess $\textbf{x}_0$ and $\textbf{b}$ associated with grid block $B_{i,j}$ \\
//\qquad    \textit{do in parallel with all processes}
\STATE $\textbf{r}_0 = \textbf{b}-\textbf{B}\textbf{x}_0$, $\textbf{s}_0 =0$, $\textbf{p}_0 =0$;\quad $\rho_0=1$,$\sigma_0=0$, $k=0$;
\WHILE{$k \leq k_{max}$ }
\STATE $k=k+1$;
\STATE $\textbf{r}'_{k} =\textbf{M}^{-1}\textbf{r}_{k-1}$; \label{pcg_scale1} \COMMENT{preconditioning}
\STATE $\textbf{z}_k = \textbf{B}\textbf{r}'_{k}$; \label{pcg_mat}\COMMENT {matrix-vector multiplication}
\STATE $update\_halo(\textbf{z}_{k})$; \COMMENT{boundary communication} \label{pcg_bc1}
\STATE $\tilde{\rho_k} = \textbf{r}_{k-1}^T\textbf{r}'_{k}$;\label{pcg_dot1}
\STATE $\tilde{\delta_k} = \textbf{z}_k^T\textbf{r}'_k$;\label{pcg_dot2}
\STATE $(\rho_k,\delta_k) = global\_sum(\tilde{\rho_k},\tilde{\delta_k})$;\label{pcg_global1} \COMMENT{global reduction}
\STATE $\beta_k = \rho_k / \rho_{k-1}$;\label{pcg_beta}
\STATE $\sigma_k = \delta_k - \beta_k^2\sigma_{k-1}$;\label{pcg_sigma}
\STATE $\alpha_k = \rho_k /\sigma_{k}$;\label{pcg_alpha}
\STATE $\textbf{s}_k = \textbf{r}'_{k-1} +\beta_k\textbf{s}_{k-1}$;\label{pcg_scale1}
\STATE $\textbf{p}_k = \textbf{z}_{k} +\beta_k\textbf{p}_{k-1}$;\label{pcg_scale2}
\STATE $\textbf{x}_k =\textbf{x}_{k-1} +\alpha_k \textbf{s}_k$;\label{pcg_scale3}
\STATE $\textbf{r}_k =\textbf{r}_{k-1} -\alpha_k\textbf{p}_k$;\label{pcg_scale4}
\IF{ $k \% n_{c} == 0$}
\STATE check convergence;
\ENDIF
%\STATE \textbf{if} $||\textbf{r}_k|| \le \epsilon$  \textbf{return} ;\COMMENT{check convergence every $n_c$ iterations}
%\ENDIF
\ENDWHILE
\end{algorithmic}
\end{algorithm}

For reference, the ChronGear method is provided in Algorithm
\ref{alg:pcg}.  As shown in Algorithm \ref{alg:pcg}, the ChronGear
solver mainly contains three parts: computation, boundary updating,
and global reduction.  Computation involves matrix-vector and
vector-vector multiplications and vector scaling, all of which have good scalability.  The boundary communication that is required to
update the halo area after the matrix-vector multiplication is also
non-problematic as the amount of communication required is
bounded. The most time-consuming global-reduction process is the inner
product operation shown in step \ref{pcg_global1}.

%----------------------------------------------------------------------------
\subsection{Communication Bottleneck}\label{se:bottleneck}
Assume $p=m^2$ processes are used and each process has exactly one
grid block (a reasonable choice for POP on a massively parallel machine),
the total time of the barotropic mode is equal to the execution time of the ChronGear solver on any block.
For each solver iteration, we choose $\mathcal{T}_c$, $\mathcal{T}_b$ and $\mathcal{T}_g$ to be the execution time of the computation,
boundary updating and global reduction, respectively.
%----------------------------------------------------------------------------


Algorithm \ref{alg:pcg} indicates that the computation involves four vector scaling operations in
steps \ref{pcg_scale1},  \ref{pcg_scale2} ,\ref{pcg_scale3} and  \ref{pcg_scale4},
two vector-vector multiplication operations of the inner products in steps \ref{pcg_dot1} and \ref{pcg_dot2},
and one matrix-vector multiplication operation in step \ref{pcg_mat}. 
$\mathcal{T}_c= (4 n^2 +2n^2+ 9n^2)\theta + \mathcal{T}_{p}  =15\frac{\mathcal{N}^2}{p}\theta+\mathcal{T}_{p}$,
where $\theta$ is the time unit per floating-point operation and
$\mathcal{T}_{p}$ is the execution time of preconditioning which may vary depending on the applied preconditioner.
For example, $\mathcal{T}_{p} =  \frac{\mathcal{N}^2}{p}\theta$ when a diagonal preconditioner is used. 
When the number of processes increases, $\mathcal{T}_c$ decreases and has a lower limit of zero.

Boundary updating occurs in the halo regions for each process, after operations like matrix-vector multiplication and non-diagonal preconditioning which requires one or more boundary layers. 
It is worth mentioning that only one boundary updating is needed each iteration even when a non-diagonal preconditioner is used, because every process keeps its own block and two extra halo layers in POP. 
The actual time depends on the network delay and the volume of the halo regions.
Since the halo size is $2$, the volume in each boundary is $2n$ and decreases as the number of processes increases. 
%It is worth mentioning that the preconditioning process usually requires at least one boundary layer,
%except for the diagonal preconditioner, which means that values on the outmost points are not updated after preconditioning. 
The total updating time for each iteration is then $\mathcal{T}_b =4\alpha +(4\times 2n)\beta=4\alpha +(\frac{8\mathcal{N}}{\sqrt{p}})\beta $,
where $\alpha$ is point-to-point communication latency per message and $\beta$ accounts for transfer time per byte (inverse of bandwidth). 
The updating time also decreases as the number of processes increases but has a lower bound of $4\alpha$.


There is only one global reduction in a ChronGear iteration,
thus the reduction time satisfies $\mathcal{T}_g= \log p \alpha$, assuming that a binomial tree approach is used in the reduction operation.
Obviously, $\mathcal{T}_g$ increases monotonically with the number of processes $p$.
Note that the global reduction has virtually no data exchange since there is only one number from each process.
%Let $T_0$ be the time unit of one floating-point operation and $B$ be the number of floating-point numbers transmitted by the network per second from process to process.
%Provided that the processor frequency and network bandwidth are $S_{cpu}$ and $B_{net}$, and that their efficiencies are $R_{cpu}$ and $R_{net}$, then $T_0 = R_{cpu} S_{cpu}^{-1}$, and $B = \frac{1}{8}R_{net}B_{net}$.
By considering the all three parts, the execution time of one diagonal preconditioned ChronGear solver step can be expressed as:
\begin{eqnarray}
%\begin{tabular}{l}
\label{t_pcg}
&\mathcal{T}_{cg}=\mathcal{K}_{cg} (\mathcal{T}_c + \mathcal{T}_b+\mathcal{T}_g )\nonumber \\
&=\mathcal{K}_{cg} [16 \frac{\mathcal{N}^2}{p}\theta + \frac{8\mathcal{N}}{\sqrt{p}}\beta +(4+\log p)\alpha]
%\end{tabular}
\end{eqnarray}
where $K_{cg}$ is the number of iterations in one ChronGear step which does not change with the number of processes \cite{hu2013scalable}.
Equation (\ref{t_pcg}) shows that the time required for computation and boundary updating decreases as the number of processes increases.
But the time required for global reduction increases with increasing
numbers of processes. Therefore, we expect the execution time of the
ChronGear solver to increase when the number of processors exceeds a certain threshold.
Indeed, the scaling behavior of ChronGear in the 0.1\degree POP is consistent with the above analysis (Fig.\ref{fig:ChronGearCOMP}).
The execution time of global reduction becomes dominant and increases when more than 2,634 cores are used.


\begin{figure}[!t]
\vspace{-10pt}
\begin{center}
	\includegraphics[width=1.0\linewidth]{ChronGear_comp.eps}
\end{center}
\caption[] {Time components of ChronGear Solver in 0.1\degree POP}
\label{fig:ChronGearCOMP}
\end{figure}
%However,it still inherits the poor scalability from PCG.  
%We tested the diagonal preconditioned ChronGear solver in  0.1\degree POP, and found that the scaling behavior is consistent with the above analysis. As shown in Fig.\ref{fig:ChronGearCOMP}, the execution time of global reduction becomes dominates in the ChronGear solver and increases when more than 2634 cores are used.


\section{P-CSI Solver} \label{se:psi}
%----------------------------------------------------------------------------
To address the scalability bottleneck in POP, a barotropic solver that
requires as few global reductions as possible is desired.
%Originally less efficient methods, such as Chebyshev iteration, were reconsidered in POP. 
%Chebyshev iteration was revisited by Gutknecht \cite{gutknecht2002chebyshev} in 2002, and was identified as being suitable for massively parallel computers with high communications costs. 
In \cite{hu2013scalable},  Hu et al. developed an appropriate solver based
on Stiefel's Chebyshev iteration method (CSI) and do a preliminary
evaluation at modest core counts in a standalone version of POP.
Here, we further improve the CSI by developing an effective preconditioner and implement the optimized P-CSI solver in the POP within the CESM framework. 
%P-CSI provides interfaces for different kinds of preconditioner. 
% As early as 1985, Saad et al.\cite{saad1985solving} implemented a generalization of P-CSI on a linear array of processors and claimed that this generalization is more favorable than conjugate gradient method in some cases when the eigenvalues are known.

\subsection{Algorithm and Evaluation} \label{se:psialg}
Like CSI, P-CSI does not require inner product operations and thus eliminates the bottleneck of global reduction as in PCG and ChronGear.
Instead of requiring prior knowledge about the spectrum of the coefficient matrix $A$, P-CSI needs the spectrum of the preconditioned matrix $M^{-1}A$.
For real symmetric and positive definite matrices with preconditioners,
such as the coefficient matrix $A$ and its corresponding diagonal preconditioner $M = \Lambda(A)$ in POP,
only approximations of the largest and smallest eigenvalues $\lambda_{max}$ and $\lambda_{min}$ of $M^{-1}A$ are needed to ensure the convergence of P-CSI. 
These two extreme eigenvalues can be estimated efficiently by the Lanczos method. 
The pseudo code of the new P-CSI algorithm designed for POP is shown in Algorithm 2.

\begin{algorithm}[!t]
\caption{Preconditioned Classical Stiefel Iteration solver}
\label{alg:ppsi}
\begin{algorithmic}[1]
\REQUIRE Coefficient matrix $\textbf{B}$, preconditioner $\textbf{M}$, initial guess  $\textbf{x}_0$ and right hand side vector $\textbf{b}$ associated with grid block $B_{i,j}$; Estimated eigenvalue boundary $[\nu,\mu]$;  \\
 // \qquad    \textit{do in parallel with all processes}
\STATE $\alpha =\frac{2}{\mu -\nu}$, $ \beta = \frac{\mu +\nu}{\mu -\nu}$, $\gamma = \frac{\beta}{\alpha}$, $\omega_0 =\frac{ 2}{\gamma}$;\quad $k = 0$;
\STATE $\textbf{r}_0 = \textbf{b}-\textbf{B}\textbf{x}_0$; $\textbf{x}_1 =\textbf{x}_0 -\gamma^{-1}\textbf{M}^{-1}\textbf{r}_0$; $\textbf{r}_1 =\textbf{b} -\textbf{B}\textbf{x}_1$; 
\WHILE{$k \leq k_{max}$ }
\STATE $k=k+1$;
\STATE $\omega_k = 1/(\gamma - \frac{1}{4\alpha^2}\omega_{k-1})$; \COMMENT{the iterated function}
\STATE $\textbf{r}'_{k-1} =\textbf{M}^{-1}\textbf{r}_{k-1}$; \COMMENT{preconditioning} \
\STATE $\Delta \textbf{x}_{k} =\omega_k\textbf{r}'_{k-1}+(\gamma \omega_k-1)\Delta \textbf{x}_{k-1}$; 
\STATE $\textbf{x}_{k} =\textbf{x}_{k-1}+\Delta \textbf{x}_{k-1}$; 
\STATE $\textbf{r}_{k} =b- \textbf{B}\textbf{x}_{k}$; \COMMENT{matrix--vector multiplication}
\STATE $update\_halo(\textbf{r}_k)$; \COMMENT{boundary communication}
\IF { $k \%  n_{c} == 0$ }
\STATE check convergence;
\ENDIF
\ENDWHILE
\end{algorithmic}
\end{algorithm}

The P-CSI algorithm and its properties are similar to that of the CSI
algorithm detailed in \cite{hu2013scalable}, with the exception of an
additional preconditioner, which is described in detail in the next
section. Thus, the total computation time of each iteration in the diagonal preconditioned P-CSI solver is $T_c =\frac{12\mathcal{N}^2}{p}\theta+\mathcal{T}_p =\frac{13\mathcal{N}^2}{p}\theta$
and the total execution time for each P-CSI solver step is
\begin{eqnarray}
\label{t_psi}
\mathcal{T}_{pcsi} = \mathcal{K}_{pcsi}(\mathcal{T}_c + \mathcal{T}_b ) \nonumber \\
= \mathcal{K}_{pcsi}[13\frac{\mathcal{N}^2}{p}\theta+ 4\alpha + \frac{8\mathcal{N}}{ \sqrt{p}}\beta]
\end{eqnarray}
where $K_{pcsi}$ is the number of iterations in one P-CSI solver step.  

Note that the overall convergence rate of P-CSI will be slower than
that of the PCG or ChronGear method and, 
as a result, PCSI requires a larger number of iterations ($K_{pcsi} > K_{cg}$). 
%under the same convergence tolerances.  
This will translate into a higher execution
time for P-CSI than ChronGear at smaller core counts when global
reductions are not an issue.  However, for high-resolution grids when
many cores are required, P-CSI will be notably faster than ChronGear
per iteration (see Equations (\ref{t_pcg}) and (\ref{t_psi})), which
will result a reduction in time to convergence.

\subsection{Eigenvalue Estimation for the P-CSI}
The eigenvalue estimation method for CSI \cite{hu2013scalable} is amended for P-CSI to calculate the eigenvalues of the preconditioned matrix $M^{-1}A$.
A series of tridiagonal matrices $T_m (m=1,2,...)$ whose largest ($\lambda_{max}$) and smallest ($\lambda_{min}$) eigenvalues converge to those of $M^{-1}A$
can be constructed using the Lanczos method \cite{Paige1980235} since the actual values of $\lambda_{min}$ and $\lambda_{max}$ are difficult to obtain. 
%For reference, Lanczos-based eigenvalues estimation procedure in given in Algorithm \ref{alg:lanczos_pre}. 
A more detailed discussion as related to CSI can be found in
\cite{hu2013scalable}.
%In step \ref{lanczos_converge}, $\epsilon$ is a user-defined tolerance.
In experiments, we found that  setting the Lanczos convergence tolerance $\epsilon$ to $0.15$ works efficiently in both 1\degree and 0.1 \degree cases with different preconditioners.
%$T_m$ contains $\alpha_i (i=1,2,...,m)$ as its diagonal entries and $\beta_i (i=1,2,...,m-1)$ as off-diagonal entries listed below
%\[ T_{m} = tridiag\left(\begin{array}{ccccccc}
%&\beta_1 && \bullet & &\beta_{m-1}&    \\
%\alpha_1 & &\alpha_2 && \bullet &&\alpha_{m}\\
%&\beta_1 && \bullet & & \beta_{m-1}&
%\end{array} \right)\]
Fig.\ref{fig:iter} indicates that only a small number of Lanczos steps
are necessary to generate
eigenvalue estimates of $M^{-1}A$ that result in near-optimal P-CSI
convergence. 


%%IF WE NEED MORE SPACE WE CAN LEAVE OUT ALGORITHM 3
%convergence speed of P-CSI reaches its theoretical optimum when $\nu = \lambda_{min}$ and $\mu =\lambda_{max}$. 
%Accurate values of $\lambda_{min}$ and $\lambda_{max}$ are difficult to obtain. In addition, any transformation of the coefficient matrix $A$ is ill-advised because $A$ was distributed to processes.
%To utilize the parallelism of POP, we employ Lanczos method  to construct
%In practice, we find that this theoretical optimum has a iteration number close to the one of PCG.


%\begin{algorithm}[!ht]
%\caption{Lanczos-based Eigenvalue Estimation for Preconditioned Matrix}
%\label{alg:lanczos_pre}
%\begin{algorithmic}[1]
%\REQUIRE Coefficient matrix $\textbf{B}$, preconditioner $\textbf{M}$, and random vector $\textbf{r}_0$ associated with grid block $B_{i,j}$; \\
% //\qquad    \textit{do in parallel with all processes}
%\STATE $\textbf{s}_0=\textbf{M}^{-1}\textbf{r}_0$;\quad $\textbf{q}_1 = \textbf{r}_0/({\textbf{r}_0^T\textbf{s}_0})$;\quad $\textbf{q}_0=\textbf{0}$;
%\STATE $T_0=\emptyset$;\quad $\beta_0 =0$;\quad  $\mu_0 =0$;\quad $j=1$;
%\WHILE{$j<k_{max}$}
%\STATE $\textbf{p}_j = \textbf{M}^{-1}\textbf{q}_j$; \quad $\textbf{r}_j=\textbf{B}\textbf{p}_j-\beta_{j-1}\textbf{q}_{j-1}$;
%\STATE $update\_halo(\textbf{r}_j)$;
%\STATE $\tilde{\alpha}_j =\textbf{p}_j^T\textbf{r}_j$; \quad $\alpha_j=global\_sum(\tilde{\alpha}_j)$;
%\STATE $\textbf{r}_j=\textbf{r}_j-\alpha_{j}\textbf{q}_{j}$; \quad $\textbf{s}_j = \textbf{M}^{-1}\textbf{r}_j$; 
%\STATE $\tilde{\beta}_j = \textbf{r}_j^T\textbf{s}_j$; \quad $\beta_j=sqrt(global\_sum(\tilde{\beta}_j))$;
%\STATE \textbf{if} $\beta_j == 0$ \textbf{then} \textbf{return}
%\STATE $\mu_j = max(\mu_{j-1},\alpha_j+\beta_j+\beta_{j-1})$; \label{lan_gersh}\\
%\STATE $T_j=tri\_diag(T_{j-1},\alpha_j,\beta_j)$; \COMMENT{Tridiagonal}\label{lan_tm}
%\STATE $\nu_j = eigs(T_j,'smallest')$ ; \label{lan_nu}
%\STATE \textbf{if} $|\frac{\mu_j}{\mu_{j-1}} -1 |< \epsilon\quad\textbf{and}\quad|1- \frac{\nu_j}{\nu_{j-1}}|< \epsilon$ \textbf{then} \textbf{return}; \label{lanczos_converge}
%\STATE $\textbf{q}_{j+1}= \textbf{r}_j/\beta_j$;\quad $j=j+1$;
%\ENDWHILE
%\end{algorithmic}
%\end{algorithm}

%This algorithm assume that $\textbf{A}$ is positive defined symmetrical matrix.  However, it is simple to adjust the algorithm for negative defined matrix as in our case. Just negating both $\textbf{A}$ and its preconditioner. 

\begin {figure}[!t]
\centering
\includegraphics[width=1.0\linewidth]{solver_iteration}
\caption []{Relationship of P-CSI iterations and Lanczos steps in 1\degree POP \label {fig:iter}}
\end {figure}


\section{Error Vector Propagation (EVP) Preconditioning} \label{se:evp}
Equations (\ref{t_pcg}) and (\ref{t_psi}) show that the total execution time of the barotropic solver is the multiplication of
iteration number and the execution time per iteration.
With increasing numbers of processor cores, the execution time of computation in each iteration decreases
but the execution time of communication increases.
In order to reduce the communication cost, the preconditioning is commonly used to reduce iterations for the solver convergence
assuming the cost of preconditioning is reasonable.
Three major concerns of preconditioners for large sparse linear systems are efficiency, scalability and preprocessing costs. 
The existing ChronGear solver in POP has benefited greatly using a simple diagonal preconditioning \cite{pini1990simple, reddy2013comparison}. 
However, its convergence rate is limited and further reduction of the iteration count can significantly reduce the associated communication cost
to enhance the scalability in a large number of processors. 
The performance of both P-CSI and the existing ChronGear solvers can be further improved by a more effective preconditioner.

%The data distribution and communication cost can only tolerate lower level of factorization like ILU(0). L and U have the same nonzero structure as the lower and upper parts of A respectively, which often results in a crude approximation has a very bad effect on convergence speed. 
%More accurate ILU factorizations is not suit for ocean model, because it requires more communications and computations. 
%Also, the preprocessing cost to compute the factors is higher. 

%In 1985, Concus et al. \cite{concus1985block} used the banded approximate of the matrix inverse as a preconditioner,
%and achieved higher efficiency than other preconditioners on elliptic partial differential equations. 
%NOTE YH what is "local approximate inverse of the true operator"? 
%Smith et al. \cite{smith1992parallel} employed a polynomial preconditioning method and a local approximate-inverse method.
%POP supports a preconditioner consisting of a 9-point operator which is a local approximate-inverse of the true operator.

\subsection{Block Preconditioning}

The most widely used preconditioning techniques in sequential simulations are the incomplete factorization methods
like incomplete LU factorization (ILU) and its variants \cite{benzi2002preconditioning}.
However, they are ill-suited for parallel computers because they require a recursive operation which limits parallelization. 

Some parallelizable preconditioning methods such as polynomial preconditioning, 
approximate inverse preconditioning, multigrid preconditioning and block preconditioning 
have drawn much attention recently.
High-order polynomial preconditioning can reduce iterations as effectively as ILU, however,
its computational overhead offsets its superiority to diagonal preconditioning 
\cite{meyer1989numerical,smith1992parallel}. 
A $k$th-order polynomial preconditioner requires $k-1$ matrix vector multiplications in each iteration. 
The approximate inverse preconditioner can be highly parallelized but requires to
solve another linear system which is several times larger than the one it preconditions 
\cite{smith1992parallel,bergamaschi2007numerical}. 
This makes it even less attractive than the diagonal preconditioner.
Multigrid is another popular preconditioning method which is highly scalable and efficient in preconditioning the sparse linear systems developed from elliptic equations \cite{baker2012scaling}. 
But the cost of setting up the multigrid preconditioning is so high that it is not worth
using multigrid for iterative solvers in problems like POP, which converge in tens or hundreds of iterations. 
Block preconditioning has been commonly shown as an effective alternative in parallel computing 
\cite{concus1985block, white2011block},
which makes use of the block structure of the coefficient matrix
arising from discretization of the elliptic equations and factorizes the matrix block-by-block 
instead of point-by-point.

\begin {figure}
\centering
\includegraphics[width=0.8\linewidth]{blockpreconditioning.eps}
\caption[] {Sparsity pattern of the coefficient matrix developed from nine-point stencils. 
The whole domain is divided into $3\times3$ non-overlapping blocks.
Elements in red rectangles are coefficients between points in blocks. 
Elements in blue rectangles are coefficients between points from the $i$-th block and its neighbor blocks. \label{fig:blockprecond}}
\end{figure}

Fig \ref{fig:blockprecond} illustrates the diagram of the block precondition.
Providing that the linear system of $\mathcal{N} \times \mathcal{N}$ grids is reordered block by block with size of $n\times n$
(e.g., $\mathcal{N}/3\times \mathcal{N}/3$ in Fig. \ref{fig:blockprecond}),
the coefficient matrix $A$ can be represented by a nine-diagonal block matrix. 
Each row of this matrix contains nine sub-matrix (Fig \ref{fig:blockprecond}). %(arranged in their spatial relationship)
Each $B_i$ (red blocks) is a block matrix containing coefficients of the grids in the i-th block, which share the same structure as $A$ but has a smaller size of $n^2\times n^2$. 
$B_i^e$,$B_i^w$,$B_i^n$ and $B_i^s$ are diagonal block matrix containing coefficients of points on east, west, north and south boundaries
and the points on their respective neighboring blocks, thus having at most $3n$ nonzero elements distributed on $n$ rows. 
$B_i^{nw}$,$B_i^{ne}$,$B_i^{sw}$ and $B_i^{se}$ have only one nonzero element, representing the coefficient of corner points
and their neighboring points on the northwest, northeast, southwest and southeast blocks. 

The traditional block preconditioning method constructs the approximate inverse of $A$ by sequentially factorizing it with approximations of $B_i^{-1}$,
which is ill-suited for parallel applications. 
%the diagonal approximation or approximation from Cholesky factors of $B_i^{-1}$
On the contrary, the inverse of the block diagonal of $A$, which provides a good approximation for $A$, can be calculated naturally in parallel.
The inverse of the diagonal block matrices is  
\begin{eqnarray*}
M^{-1}=    \left [
        \begin{array}{ccccccc}
        B_1^{-1} &   &  \\
         & \ddots&  \\
        &   &  B_{m^2}^{-1} \\
    \end{array}
    \right ]
\end{eqnarray*}
Using this $M$ as a preconditioner, the preconditioning process $\textbf{x} = M^{-1}\textbf{y}$ is usually transformed into solving the sparse linear equations $B_i \textbf{x}_i = \textbf{y}_i$ for each block,
instead of explicitly constructing $B_i^{-1}$.
This parallel block preconditioning is preferable in parallel application due to less computation to solve the equations with LU decomposition
than multiplying $\textbf{y}$ by $M^{-1}$. 

\subsection{Error Vector Propagation Method}
The arithmetic complexity of solving the equations $B_i \textbf{x}_i = \textbf{y}_i$ with LU decomposition is $\mathcal{O}(n^4)$,
providing that the LU decomposition is previously initialized.
The ILU method is much more efficient only in sequential application than in parallel application
which only supports no-fill ILU that has very crude approximate solution. 
To the best of our knowledge, the most efficient alternative to solve elliptic equations is
the EVP method in terms of the iteration counts \cite{roache1995elliptic},
which marches over the whole domains based on the discretized equations with an arithmetic complexity $\mathcal{O}(n^2)$.
%EVP is an alternative of fast direct solvers for elliptic equations, 
The EVP method and its variant have been used in some ocean models (e.g., Sandia Ocean Modeling System \cite{dietrich1987ocean} and CANadian version of DIEcast \cite{sheng1998candie}).
Tseng and Chien \cite{tseng2011parallel} further employed a modified parallel EVP method based on domain-decomposition as a solver for the global ocean simulation. 

\begin {figure}[!t]
\centering
\includegraphics[width=0.8\linewidth]{evp9pmarch1.png}
\caption []{EVP marching method for nine-point stencil. The solution on point $(i+1,j+1)$ can be calculated using the equation on point $(i,j)$, providing solutions on other neighbor points of point $(i,j)$ (all magenta points).  \label {fig:evp9p}}
\end {figure}

We can discretize Equation \ref{eq:ssh} into the following form so that we can march the solution northeastward assuming all other neighboring points are exactly known (pink circles):
\begin{eqnarray}
\label{eq:evp9p}
&\eta_{i+1,j+1} = (1/A_{i,j}^{ne} )(\psi_{i,j} - A_{i,j}^0\eta_{i,j}-A_{i,j}^e\eta_{i+1,j} \nonumber\\
&-A_{i,j}^n\eta_{i,j+1}-A_{i-1,j}^{ne}\eta_{i-1,j+1} +A_{i-1,j}^e\eta_{i-1,j}\nonumber\\
&-A_{i-1,j-1}^{ne}\eta_{i-1,j-1}-A_{i,j-1}^n\eta_{i,j-1}- A_{i+1,j-1}^{ne}\eta_{i,j-1} )
\end{eqnarray}
%\textbf{I suggest you to make Fig. 5 to use a different color or label to represent the specified boundary conditions}
Fig.\ref{fig:evp9p} provides an example for a Dirichlet boundary elliptic equation $B\textbf{x} = \psi$ on a small domain. 
Let's define the interior points next to the south and west boundaries
as the initial guess points $\textbf{e}$ and those next to the north and east boundaries are the final boundary points $\textbf{f}$
(e.g., $\textbf{e}= \{E_1, \dots, E_7\}$, $\textbf{f}= \{F_1, \dots, F_7\}$ in Fig.\ref{fig:evp9p}).
If the true solution on $\textbf{e}$ is known, the exact values over the whole domain can be computed
sequentially from southwest to northeast corners, using equation \ref{eq:evp9p}. This procedure is referred as marching. 
Unfortunately, the value on $\textbf{e}$ is often not known until the elliptic equation is solved. 
However, we can get a solution $\textbf{x}$ satisfying the elliptic equation on the whole domain except on the boundary,
by first guessing the value $\textbf{x}|_\textbf{e}$ on $\textbf{e}$ and then calculating the rest using the marching method. 
Then $E=(\textbf{x} -\eta)|_\textbf{e}$ and $F=(\textbf{x} -\eta)|_\textbf{f}$ are error vectors on $\textbf{e}$ and $\textbf{f}$, respectively. 
The error vector $F$ is already known since $\textbf{f}$ are boundary points (Dirichlet boundary condition is imposed).
The relationship between the error on initial guess points and the final boundary points can be represented as $F=W*E$. 
This influence coefficient matrix $W$ can be formed by marching on the whole domain with unit vectors on the initial guess points and zero residual value
in the whole domain. 
We summarized this EVP algorithm for an elliptic equation with zero boundary in Algorithm \ref{alg:evp}. 


\begin{algorithm}[!h]
\caption{Nine-point Error Vector Propagation method}
\label{alg:evp}
\begin{algorithmic}[1]
\REQUIRE Residual $\psi$ associated with a domain containing $n\times n$ grids, $k = size(\textbf{e})=2n-5$; \\
//\qquad \textit{preprocessing }
\STATE  $\textbf{x} = \textbf{0}$
\FOR {i = 1 \TO k}
\STATE $\textbf{x}|_\textbf{e}(i) = 1$
\STATE $\textbf{x} = marching(\textbf{x},\textbf{0})$
\STATE $W(i,:) = \textbf{x}|_\textbf{f}$
\STATE $\textbf{x}|_\textbf{e}(i) = 0$
\ENDFOR 
\STATE $R = inverse(W)$ \\
//\qquad \textit{solving }
\STATE $\textbf{x}= marching(\textbf{x},\psi)$
\STATE $F = (\textbf{x} - \eta)|_\textbf{f}$
\STATE $\textbf{x}|_\textbf{e} =\textbf{x}|_\textbf{e} - R*F$
\STATE $\textbf{x} = marching(\textbf{x},\psi)$
\end{algorithmic}
\end{algorithm}

The EVP method contains two processes: preprocessing and solving. In the preprocessing step, the influence coefficient matrix
and its inverse are computed, involving a calculation of  $\mathcal{C}_{pre}= ((2n-5)* 9n^2 + (2n-5)^3 = \mathcal {O} (26n^3)$.
The solution is then directly solved in the solving step with a calculation of $\mathcal{C}_{evp}= 2* 9n^2 + (2n-5)^2 = \mathcal{O} (22n^2)$. 
The estimation of operation counts indicates that the EVP is more efficient than other direct solvers like LU decomposition
or iterative methods such as PCG for elliptic equations, if only the solving step is considered.
This makes the EVP practical in real application since the preprocessing is only needed once at the beginning
to obtain the influence coefficient matrix and its inverse.

\subsection{EVP as A Parallel Preconditioner}
The EVP method described in Algorithm \ref{alg:evp} provides an alternative to efficiently solve the elliptic equations.
%thus making it possible to use the block diagonal of $A$ as a preconditioner. 
%\textbf{not sure what the above sentence means mean EVP combined with block diagonal?}
Here, we develop a block preconditioning technique based on the EVP method in each block to further enhance
the performance of the barotropic solver in POP.
Each preconditioning solves elliptic equations $B_i \textbf{x} = \textbf{y}  (i=1,...,m^2)$ in parallel. 
%We use the original version of EVP as the local preconditioner for simplicity and efficiency. 
%We developed a new preconditioning method inspired by the block preconditioning. 

A major drawback of the EVP method is that it cannot solve any large domain problem without further modification due to
its numerical instability in marching \cite{roache1995elliptic}. 
However, the EVP can solve any small domain up to the size of $12\times 12$ with
an acceptable round-off error of $10^{-8}$ when double-precision floating-point format is used, which makes it ideal for the block preconditioning
in the parallel computing.
%\textbf{are we using single-precision? or double? I think we have to use double, right?} %should there be proof for this? 
%Even though EVP method can be adapted to solve elliptic equations on irregular geometries. 
%Thus, we construct preconditioning matrices with small blocks. 
\begin {figure}[!t]
\centering
\includegraphics[width=1.0\linewidth]{iteration.eps}
\caption[] {Average iteration number of different barotropic solvers. \label{fig:iteration}}
\end{figure}


Therefore, the drawback of the EVP becomes a great advantage in the large scale parallel computing on the other hand
(larger number of processors facilitate smaller domains).
Furthermore, we find that the coefficients related to north, south, east and west neighbors on every point are one magnitude order smaller than the others. 
Removing these coefficients reduces the cost of calculating the EVP preconditioning matrix so that the preconditioning process
can be reduced to about a half without any significant impact on the iteration number for both ChronGear and P-CSI. 
As a result, the execution time of EVP preconditioning can be expressed as $\mathcal{T'}_{p} = 14n^2\theta= 14\frac{\mathcal{N}^2}{p}\theta$. 
The actual cost of $\mathcal{T'}_{p}$ depends on the size of the local block, thus decreasing as more processor cores are used.
%In the current version of POP, using non-diagonal preconditioning methods requires an additional boundary communication after the preconditioning in each iteration. 
%However, we find that the later boundary communication can be skipped by utilizing the fact that the halo size is 2. 
Thus, the total execution time for one ChronGear and P-CSI solver step are 
\begin{eqnarray}
\label{t_evppcg}
&\mathcal{T'}_{cg}=\mathcal{K'}_{cg} (\mathcal{T'}_c + \mathcal{T'}_b+\mathcal{T'}_g )\nonumber \\
&=\mathcal{K'}_{cg} [29 \frac{\mathcal{N}^2}{p}\theta + \frac{8\mathcal{N}}{\sqrt{p}}\beta +(4+\log p)\alpha]
\end{eqnarray}
\begin{eqnarray}
\label{t_evppsi}
\mathcal{T'}_{pcsi} = \mathcal{K'}_{pcsi}(\mathcal{T'}_c + \mathcal{T'}_b ) \nonumber \\
= \mathcal{K'}_{pcsi}[26\frac{\mathcal{N}^2}{p}\theta+ 4\alpha + \frac{8\mathcal{N}}{ \sqrt{p}}\beta]
\end{eqnarray}


%\begin{table}
%\centering
%\caption{Average iteration number of different barotropic solvers. \label{tab:iteration_evp}}
%\begin{tabular}{|l|c|c|c|c|}
%%\toprule
%\hline
%Resolution & \multicolumn{2}{|c|}{1\degree} & \multicolumn{2}{|c|}{0.1\degree}\\ \hline
%Solvers& ChronGear  & P-CSI & ChronGear	 & P-CSI  \\\hline
%NONE&  540  &570	& 150	 & 160  \\\hline
%DIAG&  240  &300	& 100	 & 120 \\\hline
%EVP &  160  &200	& 50	 & 60  \\\hline
%%\bottomrule
%\end{tabular}
%\end{table}
The implementation of EVP preconditioning in POP significantly reduces the iterations in both ChronGear and P-CSI solvers. 
As shown in Fig. \ref{fig:iteration}, EVP preconditioning reduces  iterations to about one third in both 1\degree and 0.1 \degree cases, which is comparable to the approximate-inverse preconditioner proposed in \cite{smith1992parallel}. 
Although the EVP preconditioning doubles the computation in each iteration, it halves both global and boundary communication
which dominates in the barotropic execution time when large numbers of processor cores are used.
One advantage of EVP preconditioning is the low cost of solving the preconditioning matrix. 
%EVP preconditioning matrix is totally parallelized, while traditional decomposition based preconditioning requires serial processing. 
In 0.1\degree case, the cost of setting up the preconditioning matrix is less than the one of calling the solver once when 512 processor cores are used. 
This cost can be further decreased when more processors are used. 

\section{Experiments} \label{se:exp}
%----------------------------------------------------------------------------
%\subsection{Experimental Platform} \label{se:plat}

The performance of our new barotropic solver is tested in CESM1.2.0 on the Yellowstone Supercomputer at NCAR-Wyoming Supercomputing Center (NWSC) in Cheyenne, Wyoming,USA.  
Yellowstone contains 72,576 Intel Sandy Bridge processors, which are connected by a 13.6 GBps InfiniBand network. Processors are 2.6-GHz Intel Xeon E5-2670 with Advanced Vector Extensions (AVX). 

We run CESM1.2.0 for several days without disk I/O to test the solver scalability. 
To focus on the performance of POP,  we use the G\_NORMAL\_YEAR component set which couples ocean and ice with COREv2 normal year forcing.  
Two grid resolutions for POP are examined:
\begin{itemize} 
\item 1\degree resolution horizontal grid ($320\times 384$)
\item 0.1\degree resolution horizontal grid ($3600\times 2400$)
\end{itemize}
The suggested space-filling curve method for distributing blocks across processors is used in 0.1\degree case. 
%\begin{center}
%\begin{tabular}{|c|c|c|c|c|c|c|}
%\hline
%Platform & Processor &Speed    & Cores & Memory & Cache & Network \\
%\hline
%TS100 & Intel Xeon  & 2.93GHz & 12   &    24/48G   & 12M & InfiniBand QDR \\
%\hline
%SW  	  & SW1600    & 1.1GHz   & 16   & 16G       & N/A  & InfiniBand QDR\\
%\hline
%Yellowstone  & Intel Xeon    & 2.6GHz   & 16   & 32G       & 20M  & InfiniBand FDR \\
%\hline
%\end{tabular}
%\end{center}

\begin {figure}[!t]
\centering
\includegraphics[scale=0.5]{1deg_solverruntime}
\caption []{Runtime of barotropic solvers in the 1\degree POP for five days.\label {fig:runtime1}}
\end {figure}


The execution time of the barotropic solvers in 1\degree POP on a series of cores is shown in Fig.\ref{fig:runtime1}. 
P-CSI out-performs ChronGear on different processor cores and reduces the solver execution time from 0.60s to 0.44s per simulation day, that is a 1.4X speedup. 
EVP preconditioning provides speedup for both ChronGear and P-CSI. In the case of 384 cores, EVP preconditioner achieves a speed of 0.33s per simulation day which is 1.8X of the original ChronGear solver. 

A similar scenario happens in the case of 0.1\degree POP. 
As shown in Fig.\ref{fig:runtime01}, ChronGear stops scaling after 2,634 cores. While P-CSI scales well until 4,028 cores are used. 
The implementation of P-CSI in the 0.1\degree resolution POP accelerates the barotropic mode by 4.7 times, from 12.8s to 2.7s per simulation day, on 9,128 cores. 




%\begin{table}
%\centering
%\caption{Configuration of two CESM cases  used. \label{tab:caseinfo}}
%\begin{tabular}{|l|r|r|r|}
%%\toprule
%\hline
%Configure&  compset  & Ocn Grid &  Run type\\\hline
%1\degree & G &gx1v6(320x384) & hybrid \\\hline
%0.1\degree  &  G &tx0.1v2(3600x2400)  &startup \\\hline
%%\bottomrule
%\end{tabular}
%\end{table}


%Fig.1(b) shows runtime ratio of barotropic and baroclinic modes of POP. Baroclinic mode deals with a three dimensional dynamical process which occupies a major proportion of computation of POP. It dominates in POP when process is few and computation time is much larger than communication. However, when using thousands of cores, runtime of baroclinic parts decreases linearly while on the opposite, runtime of barotropic increases because of global reduction.
As talked about in section \ref{se:baro}, execution time of the barotropic solver takes an increasing percentage of the whole model. 
Thus improvement in barotropic mode will benefit the speed of the whole POP model, especially when large numbers of processor cores are used. 
\begin{table}[!h]
%\vspace{-10pt}
\centering
\caption{Percentage of improvement of the execution time in 1\degree POP. \label{tab:improve_1}}
\begin{tabular}{|l|r|r|r|r|}
%\toprule
\hline
Number of cores & 48  & 96  & 192& 384 \\\hline
ChronGear+EVP &0.1\% &0.9\%	&5.8\%&7.9\% \\\hline
P-CSI+Diagonal  & 1.5\% &2.4\%&7.9\%  &7.9\% \\\hline
P-CSI+EVP	      &0.3\%& 2.3\%	&10.0\%  &13.0\% \\\hline
%\bottomrule
\end{tabular}
\end{table}

\begin {figure}[!t]
\centering
\includegraphics[scale=0.5]{01deg_solverruntime}
\caption []{Runtime of barotropic solvers in the 0.1\degree POP for one day.\label {fig:runtime01}}
\end {figure}
Detailed improvement of the whole model compared with the original diagonal preconditioned ChronGear solver would be found in Table \ref{tab:improve_1}. The combination of P-CSI and EVP preconditioner improves the whole model by 13.0\% on 384 processor cores. 
Detailed improvement of the 0.1 \degree POP  would be found in Table \ref{tab:improve_01}. 
The combination of P-CSI and EVP preconditioner improves the whole model by 25.7\% on 9128 processor cores. 
EVP preconditioning contributes 7.8\% improvement to the original ChronGear solver, as well as 3.4\% extra improvement to P-CSI solver. 



\begin{figure*}[!t]
\begin{center}
	\includegraphics[width=1.0\linewidth]{01solvercomp_withoutbound.eps}
\end{center}
\caption[] {Time components of barotropic solvers in 0.1\degree POP. Four kinds of solvers are ChronGear or P-CSI with a diagonal preconditioner or an EVP preconditioner. (a) Execution time of global reduction, (b) Execution time of boundary communication, (c) Execution time of computing. }
\label{fig:component}
\end{figure*}

\begin{table*}
\centering
\caption {Percentage of improvement of the total execution time in 0.1\degree POP. \label{tab:improve_01}}
\begin{tabular}{|l|r|r|r|r|r|r|r|r|}
%\toprule
\hline
Number of cores & 512  & 1224   & 2634 & 3476 & 4028 & 6124 & 9128 & 12330\\
\hline
ChronGear+Evp &-0.3\% &0.2\%&3.4\%  & 6.2\%& 8.3\% & 10.7\%& 12.4\% & 14.0\%\\\hline
P-CSI+Diagonal  & 1.3\% & 2.1\%	&7.5\%  &12.1\% &15.6\% &23.2\% &22.3\% &30.5\%\\\hline
P-CSI+Evp	      &-0.2\% & 1.1\%&6.5\%  &12.2\% &16.5\% &26.0\% &25.7\% &35.0\%\\
\hline
%\bottomrule
\end{tabular}
\end{table*}

\begin{table*}
%\vspace{-10pt}
\centering
\caption {Ideal simulation rate of 0.1\degree POP \label{tab:improve_01}}
\begin{tabular}{|l|r|r|r|r|r|r|r|r|}
%\toprule
\hline
Number of cores & 512  & 1224   & 2634 & 3476 & 4028 & 6124 & 9128 & 12330\\
\hline
ChronGear     &0.71 &1.69&3.31  &3.83 &4.28 &4.72&6.02 &6.33\\\hline
ChronGear+Evp &0.70 &1.68&3.42  &3.99 &4.53 &5.08&6.69 &7.34\\\hline
P-CSI+Diagonal  &0.72 &1.72&3.58  &4.36 &5.07 &6.15&7.75& 9.11\\\hline
P-CSI+Evp	      &0.71 &1.70&3.54  &4.36 &5.12 &6.38&8.10 & 9.74\\
\hline
%\bottomrule
\end{tabular}
\end{table*}
Simulation rate (simulated years per wall-clock day) is a popular criterion for model performance. Here we use the ideal simulation rate (only the execution time of STEP function in POP counts) in order to get rid of the effect from initialization, preprocessing  and disk I/O. To satisfy a rate of 5 simulated years per wall-clock day, which is the minimum necessary simulation rate to run long term climate simulations, ChronGear requires more than 6124 cores while P-CSI requires only 4028 cores. 
The EVP preconditioned P-CSI solver improves the ideal simulation rate of POP from 6.02 to 8.10 simulated years per wall-clock day. 

To understand the source of improvement, more detail about the barotropic solver is provided in Fig.\ref{fig:component}.
Fig.\ref{fig:component} tells that P-CSI outperforms ChronGear mainly because of less global reduction. 
Even though EVP preconditioning almost doubles the execution time of computation, it helps to reduce both boundary and global communication. 
%Exception happens when small number of cores are used, due to load imbalance. 
\begin {figure}
%\vspace{-5pt}
\centering
\includegraphics[width=1.0\linewidth]{POPStepComp_pcsi.eps}
\caption[] {Percentage of execution time in 0.1\degree POP using the P-CSI solver.\label{fig:StepComp_pcsi}}
%\vspace{-10pt}
\end{figure}
It is interesting that when less than 4,000 cores are used, the execution time of global reduction and boundary communication does not decrease after replacing the diagonal preconditioner with EVP preconditioner for P-CSI. 
This is mainly due to the load imbalance of POP. 
The less the number of cores are used, the bigger the block size on each process is. 
Because of the existence of land, bigger block size leads to more imbalance between blocks.   
This explains why the execution time of global reduction decreases when less than 1,224 cores are used, which conflicts with the theoretical result in equation \ref{t_pcg} and \ref{t_psi}. 
EVP preconditioning enlarges this load imbalance because it almost doubles the execution time of computation.
Thus, even though the iteration number is reduced to a half in the case of EVP preconditioned P-CSI, each communication takes longer to complete 
 
%Even though P-CSI takes more time in computation and updating due to a larger number of iteration in each step, it outperforms ChronGear because of less global reduction.

\section{Verification} \label{se:ver}
%To ensure that P-CSI will not introduce inaccuracies into POP, we conducted an experiment with the 1\degree POP. 
%Porting validation of the current CESM is subjective and one-sided\cite{vertenstein2011cesm1}.
%It provides basically two way. 
%First, users run a set of cases on their own machine, plot the final results and compare them to the standard diagnostics plot. 
%The second one is to compare difference of several variables produced on local machine and standard machines. 
%CESM utilizes the Root Mean Square Error to measure the difference.
%\begin{figure}[!ht]
%\begin{center}
%\vspace{0pt}
%\includegraphics[width=1\linewidth,height=1\linewidth, trim={{0.0\linewidth} {0.0\linewidth} {0.4\linewidth} {1.2\linewidth}}, clip]{high-res.png}
%\end{center}
%\vspace{-20pt}
%\caption[] {Sea surface potential temperature.(will use SST error ) }
%
%\label{fig:ssT}
%\end{figure}

There is no direct verification tool for new solvers  in CESM POP currently, but it provides a way to facilitate the evaluation of a successful port on new machines. That is, to run a specific case on the new machine for five days, then compute the root-mean-square (RMS) difference of SSH field between the solution on local machine and those released as standard dataset by NCAR.



This procedure provides a criterion for validating CESM results on new machines which may contain error associated with machines and compilers. 
However, it fails to measure the error due to changes in solvers. 
We run the 1 \degree case for three years with different convergence tolerances varying from $10^{-10}$ to $10^{-16}$ in the barotropic solver. 
Define RMSE as the RMS difference between a given case and the more strict tolerance case. Only the open seas are considered here to filter out several marginal seas where POP does not simulate well. 
As shown in Fig.\ref{fig:ssh_rmse_t}, temperature RMSE of cases with different convergence tolerances do not reveal the correct order of error from solver. 
The cases with a tolerance of $10^{-10}$  and $10^{-11}$ should have bigger RMSE than others. However, during the twelfth month and twentieth months, the $10^{-10}$ case has almost the smallest RMSE. In the last two months, the case $10^{-11}$ has a smaller RMSE than all other cases except the $10^{-10}$ case, which also conflicts its order in view of solver error. 

RMSE reveals the difference between two cases. However, it is not a good criterion for new algorithms because it does not take into consideration of the chaotic nature of climate models.  In order to make the chaotic nature accounted, we employ the methodology as proposed in CESM atmosphere component CAM by Baker et al. \cite{baker2014methodology}. First we conduct an ensemble of runs which are identical to the original case except for a $10^{-14}$ perturbation in the initial temperature. This perturbation is a reasonable round-off error which climate model should be able to tolerate. 
\begin{figure}[!t]
\begin{center}
%\vspace{-20pt}
%\includegraphics[width=1\linewidth, trim={{0.2\linewidth} {0.8\linewidth} {1.2\linewidth} {1.4\linewidth}}, clip]{TEMP_RMSE.png}
\includegraphics[width=1\linewidth]{TEMP_RMSE.png}
\end{center}
%\vspace{-10pt}
\caption[] {Monthly Temperature Root Mean Square Error of different tolerance cases.}
\label{fig:ssh_rmse_t}

\end{figure}

Assume the output of the ensemble at time $T$ is $\mathcal{E}=\{X_1,X_2,...,X_m\}$, where 
$m$ is the size of the ensemble. 
For a given point $j$, we have a series of possible results from the ensemble $\{X_1(j),X_2(j),...,X_m(j)\}$.
As the ensemble size increases, this series reflects more correctly the distribution of reasonable result at the given point. 
Define the mean of this series $$ \mu (j) = \frac{1}{m}\sum_{i=1}^m X_i(j) $$
The standard deviation of this series  $$ \delta (j) = \sqrt{\frac{1}{m} \sum_{i=1}^m (X_i(j)-\mu(j))^2 }$$
The combination of $\mu$ and $\delta$ provides a criterion to test whether an additional case is close to the ensemble or not. 
Set the additional case has the result $\tilde{x}$, define the root-mean-square Z-score 
$$ RMSZ(\tilde{X}, \mathcal{E}) =  \sqrt{\frac{1}{n}\sum_{j=1}^n(\frac{\tilde{X}(j) -\mu (j)}{\delta (j)})^2}$$


Different tolerance cases are re-evaluated using RMSZ. 
As shown in Fig.\ref{fig:ssh_rmsz_t}, the ensemble cases provides a criterion to judge whether the case  is consistent with the them or not. Also, for those cases which depart so far away from the ensemble, such as the cases with the first and second largest tolerance, their RMSZ is in the same order as the order of error they introduced to the solver.  
\begin{figure}[!t]
\begin{center}
%\includegraphics[width=1\linewidth, trim={{0.2\linewidth} {0.8\linewidth} {1.2\linewidth} {1.4\linewidth}}]{TEMP-RMSZ-ensemble40-month-tol-opensea.png}
\includegraphics[width=1\linewidth]{TEMP-RMSZ-ensemble40-month-tol-opensea.png}

\end{center}
\vspace{-10pt}
\caption[] {Monthly TEMP Root Mean Square Z-score of different tolerance cases based on an ensemble of 40 members.}
\label{fig:ssh_rmsz_t}
\end{figure}
%So the error introduced by replacing ChronGear with PCG is in the same magnitude order of improving the convergence tolerance from $10^{-13}$ to $10^{-14}$ or $10^{-16}$. 

\section{Related work} \label{se:rel}
%----------------------------------------------------------------------------
%improving barotropic

Much work has been done on optimizing the performance of solving the elliptic problem required by the implicit free-surface method in ocean models.  
Most of them fall into two directions.  
The first is related to decreasing the negative effects of the global communication required by PCG or ChronGear methods.  In ocean models, OpenMP parallelism and land elimination are common strategies to reduce the number of processes and the associated  global communication overhead. Worley et al. \cite{Worley:2011:PCE:2063384.2063457} strongly recommended the OpenMP strategy to reduce the number of processes when a large count of cores are needed for the baroclinic mode.
Dennis \cite{dennis2007inverse,dennis2008scaling} proposed a load-balancing strategy based on the space-filling curve partitioning algorithms to eliminate land blocks.  This strategy not only reduces the number of processes but also leads to a better load-balance.
It doubles the simulation rate on approximately 30,000 processors.    
%MOVE TO RELATED WORK
There are currently different alternatives to mitigate the poor behavior of the PCG type of solver in the massive parallelization.
Some approaches attempt to overlap the communication with computation time\cite{beare1997optimisation}.
Some use the land elimination and load-balance strategies \cite{dennis2007inverse, dennis2008scaling}
to reduce the number of processes and the associated overhead of global reduction.
Reducing the frequency of communication also attenuates the overhead in the barotropic mode.
As early as 1997,  Beare \cite{beare1997optimisation} proposed the performance of parallel ocean general circulation models can be improved by increasing the number of extra halos and overlapping the communications with the computation.
Although these approaches may improve performance, they do not eliminate the major bottleneck of the global reduction.

%A variant of the standard conjugate gradient method presented by D'Azevedo \cite{dAzevedo1999lapack}, called the Chronopoulos-Gear algorithm, proposed a way to halve the global communication in PCG.  It combines the two separate global reductions into a single global reduction vector by rearranging the conjugate gradient computation procedure, and achieves a one third latency reduction in POP.
Another way to attenuate the bottleneck of the barotropic solver is preconditioning.
Preconditioning has been highlighted in the CG method since the 1990s. Many linear systems converge after a few PCG iterations with a suitable preconditioner.  
However, many of the most effective preconditioning techniques, such as Incomplete Cholesky decomposition and incomplete LU decomposition, are not so effective in ocean models.  
Parallelizable and special designed preconditioners are required for elliptic equations in ocean models. 
In 1985, Concus et al. \cite{concus1985block} used the banded approximate of the matrix inverse to precondition CG method on elliptic partial differential equations and achieved higher efficiency than other universal preconditioning method. 
Smith et al. \cite{smith1992parallel} employed polynomial preconditioning methods and a local approximate-inverse preconditioning method to accelerate the convergence of CG method in a parallel ocean general circulation model. 
Adamidis et al. \cite{adamidis2011high} implemented an incomplete Cholesky preconditioner in the global ocean/sea-ice model MPIOM to improve the scalability and performance of PCG.
%Watanabe \cite{Watanabe2006pcg}  used  PCG combined with an overlapping domain decomposition method to improve the convergence and reduce the communications cost between the processor elements.

This paper represents a method which improves the barotropic solver by both strategies mentioned above.  P-CSI solver eliminates the global communication which is required by CG like solvers. In the meantime, it supports an EVP preconditioner which accelerates the convergence with an efficiency comparable to other preconditioning methods developed for ocean models. 

%The improvement of the methods described above is limited due to the inherent poor data locality and sequential execution of PCG. 
%Some work has been done to accelerate the PCG solver by employing the developing hybrid  accelerating devices, such as GPUs \cite{cuomo2012pcg} and FPGAs \cite{Shida2007}.
%Cuomo et al. \cite{cuomo2012pcg} introduced the sparse approximate inverses preconditioning method into the numerical global circulation ocean model and implemented it on a GPU using a scientific computing code library.
%Shida et al. \cite{Shida2007} moved the barotropic mode onto FPGAs, and found comparative performance on 100MHz FPGAs as on GHz processors with the appropriate use of internal memory and streaming DMA.
%GPUs and FPGAs are helpful in reducing the global overhead. These devices have stronger computational ability and more memory than common CPU, so fewer devices and less communication are needed for the same scale computing job.





%----------------------------------------------------------------------------
\section{Conclusion} \label{se:conc}
A scalable and preconditioning supported barotropic solver P-CSI is implemented in POP, the ocean component of the CESM.
The EVP method is used as a preconditioner for both ChronGear and P-CSI, which outperforms diagonal preconditioning in 0.1\degree resolution POP when more than 4,028 cores are used. 
Even though the new barotropic solvers have more benefit in high resolution simulations,
it also contributes to the low resolution simulations. 
The P-CSI solver is verified to maintain a stable ocean climate within a small tolerance using an ensemble based statistical method.
In closing, this paper highlights a scalable and robust barotropic solver for free-surface ocean models.

%ACKNOWLEDGMENTS are optional
\section{Acknowledgments}
This research used computing resources provided by the Climate Simulation Laboratory at NCAR's Computational and Information Systems Laboratory (CISL), sponsored by the National Science Foundation and other agencies.

This research used resources of the National Energy Research Scientific Computing Center, a DOE Office of Science User Facility supported by the Office of Science of the U.S. Department of Energy under Contract No. DE-AC02-05CH11231.
%Generated by bibtex from your ~.bib file.  Run latex,
%then bibtex, then latex twice (to resolve references)
%to create the ~.bbl file.  Insert that ~.bbl file into
%the .tex source file and comment out
%the command \texttt{{\char'134}thebibliography}.
\bibliographystyle{abbrv}
\bibliography{hycs}  % sigproc.bib is the name of the Bibliography in this case
% This next section command marks the start of
% Appendix B, and does not continue the present hierarchy
%\section{More Help for the Hardy}
%The sig-alternate.cls file itself is chock-full of succinct
%and helpful comments.  If you consider yourself a moderately
%experienced to expert user of \LaTeX, you may find reading
%it useful but please remember not to change it.
%\balancecolumns % GM June 2007
% That's all folks!
\end{document}
