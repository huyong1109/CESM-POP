\section{Barotropic Solver} \label{se:baro}
%----------------------------------------------------------------------------
POP solves three-dimensional primitive equations with hydrostatic and Boussinesq approximations.
The time integration separates the dynamics into baroclinic and barotropic modes.
The baroclinic mode describes the three-dimensional dynamic process while
the barotropic mode solves the vertically-integrated momentum and continuity equations \cite{smith2010parallel}.

The most time-consuming portion of the barotropic simulation is the solution of the elliptic system for the sea surface height (SSH)
due to the implicit free-surface algorithm\cite{pop05}. 

The implicit elliptic equations of SSH in POP can be expressed as follows:
\begin{equation}
\label{eq:ssh}
[\nabla \cdot H\nabla  -\phi(\tau)]\eta^{n+1} = \psi(\eta^n,\eta^{n-1},\tau)
\end{equation}
where $H$ is the depth of the ocean, $\tau$ is the time step  and $\eta^n$ is the SSH at the n-th time step, and $\psi$ represents a function of previous states.
Equation (\ref{eq:ssh}) is then discretized on a two-dimensional grid using a nine-point stencil in POP. The stencil can be reorganized into a linear system $Ax =b$.
%$A_{i,j}^0$, $A_{i,j}^n$, $A_{i,j}^e$ and $A_{i,j}^{ne}$ are symmetrical coefficients between grid point $(i,j)$ and its neighbors, determined by $H$, $\tau$ and the grid lengths. The stencil confined to grid point $(i,j)$ is
%\begin{eqnarray}
%\label{eq:sten}
%&A_{i,j}^0\eta_{i,j}+A_{i,j}^e\eta_{i+1,j}+A_{i,j}^n\eta_{i,j+1} +A_{i,j}^{ne}\eta_{i+1,j+1}\nonumber\\
%&+A_{i-1,j}^{ne}\eta_{i-1,j+1} +A_{i-1,j}^e\eta_{i-1,j}+A_{i-1,j-1}^{ne}\eta_{i-1,j-1}\nonumber\\
%&+A_{i,j-1}^n\eta_{i,j-1}+ A_{i+1,j-1}^{ne}\eta_{i,j-1}=\psi_{i,j}
%\end{eqnarray}

For massive parallelism, POP divides the global domain into blocks and distributes them to processes. 
Each process only computes the evolution procedures related to the grids in its own block, and maintains a halo region to update data with its neighbors.


For simplicity, we can assume that the global domain size is $\mathcal{N}\times \mathcal{N}$,
which can be divided into $m\times m$ blocks with size of $n\times n$ ($n=\mathcal{N}/m$). 
Let $B$ and $\tilde{x}$ be the coefficient matrix and vector associated with a given block (i.e., a sub-matrix of $A$ with a size of $n^2\times n^2$),
then the matrix-vector multiplication of $B\tilde{x}$ has $9n^2$ multiplication operations \cite{hu2013scalable}.
%----------------------------------------------------------------------------
\subsection{ChronGear Solver }
ChronGear \cite{dAzevedo1999lapack} is a modified conjugate gradient method.
%with a diagonal preconditioner $M = \Lambda(A)$.
It combines two global reductions into one, which achieves a one-third
latency reduction when solving the linear system in POP that
approximates the barotropic mode.

However, as previously mentioned, when thousands of cores are used in high-resolution (0.1\degree) POP,
the global reduction needed by the inner product in ChronGear becomes a major bottleneck.
This decrease in performance is demonstrated in
Fig.\ref{fig:StepComp} for  0.1\degree POP.  The execution time for the linear solve increases as the number of processor cores increases. 
When 512 cores are used, the execution time of the barotropic solver is only 4\% of the total execution of POP, which is much smaller than the baroclinic mode time (90\%). 
However, when a few thousand cores are used, the baroclinic execution
time appropriately decreases, while the barotropic solver time
increases and exhibits poor scalability. 
When over six thousand cores are used, the percentage of the
barotropic solver execution time reaches 32\%, and the baroclinic mode
has been reduced to 50\%.   This trend continues at higher core counts.

\begin {figure}[!t]
\vspace{-5pt}
\centering
\includegraphics[scale=0.5]{POPStepComp}
\caption[] {Percentage of execution time in 0.1\degree POP using the default ChronGear solver.\label{fig:StepComp}}
\end{figure}


\begin{algorithm}[!h]
\caption{Chronopoulos-Gear Solver}
\label{alg:pcg}
\begin{algorithmic}[1]
\REQUIRE Coefficient matrix $\textbf{B}$, preconditioner $\textbf{M}$, initial guess $\textbf{x}_0$ and $\textbf{b}$ associated with grid block $B_{i,j}$ \\
//\qquad    \textit{do in parallel with all processes}
\STATE $\textbf{r}_0 = \textbf{b}-\textbf{B}\textbf{x}_0$, $\textbf{s}_0 =0$, $\textbf{p}_0 =0$;\quad $\rho_0=1$,$\sigma_0=0$, $k=0$;
\WHILE{$k \leq k_{max}$ }
\STATE $k=k+1$;
\STATE $\textbf{r}'_{k} =\textbf{M}^{-1}\textbf{r}_{k-1}$; \label{pcg_scale1} \COMMENT{preconditioning}
\STATE $\textbf{z}_k = \textbf{B}\textbf{r}'_{k}$; \label{pcg_mat}\COMMENT {matrix-vector multiplication}
\STATE $update\_halo(\textbf{z}_{k})$; \COMMENT{boundary communication} \label{pcg_bc1}
\STATE $\tilde{\rho_k} = \textbf{r}_{k-1}^T\textbf{r}'_{k}$;\label{pcg_dot1}
\STATE $\tilde{\delta_k} = \textbf{z}_k^T\textbf{r}'_k$;\label{pcg_dot2}
\STATE $(\rho_k,\delta_k) = global\_sum(\tilde{\rho_k},\tilde{\delta_k})$;\label{pcg_global1} \COMMENT{global reduction}
\STATE $\beta_k = \rho_k / \rho_{k-1}$;\label{pcg_beta}
\STATE $\sigma_k = \delta_k - \beta_k^2\sigma_{k-1}$;\label{pcg_sigma}
\STATE $\alpha_k = \rho_k /\sigma_{k}$;\label{pcg_alpha}
\STATE $\textbf{s}_k = \textbf{r}'_{k-1} +\beta_k\textbf{s}_{k-1}$;\label{pcg_scale1}
\STATE $\textbf{p}_k = \textbf{z}_{k} +\beta_k\textbf{p}_{k-1}$;\label{pcg_scale2}
\STATE $\textbf{x}_k =\textbf{x}_{k-1} +\alpha_k \textbf{s}_k$;\label{pcg_scale3}
\STATE $\textbf{r}_k =\textbf{r}_{k-1} -\alpha_k\textbf{p}_k$;\label{pcg_scale4}
\IF{ $k \% n_{c} == 0$}
\STATE check convergence;
\ENDIF
%\STATE \textbf{if} $||\textbf{r}_k|| \le \epsilon$  \textbf{return} ;\COMMENT{check convergence every $n_c$ iterations}
%\ENDIF
\ENDWHILE
\end{algorithmic}
\end{algorithm}

For reference, the ChronGear method is provided in Algorithm
\ref{alg:pcg}.  As shown in Algorithm \ref{alg:pcg}, the ChronGear
solver mainly contains three parts: computation, boundary updating,
and global reduction.  Computation involves matrix-vector and
vector-vector multiplications and vector scaling, all of which have good scalability.  The boundary communication that is required to
update the halo area after the matrix-vector multiplication is also
non-problematic as the amount of communication required is
bounded. The most time-consuming global-reduction process is the inner
product operation shown in step \ref{pcg_global1}.

%----------------------------------------------------------------------------
\subsection{Communication Bottleneck}\label{se:bottleneck}
Assume $p=m^2$ processes are used and each process has exactly one
grid block (a reasonable choice for POP on a massively parallel machine),
the total time of the barotropic mode is equal to the execution time of the ChronGear solver on any block.
For each solver iteration, we choose $\mathcal{T}_c$, $\mathcal{T}_b$ and $\mathcal{T}_g$ to be the execution time of the computation,
boundary updating and global reduction, respectively.
%----------------------------------------------------------------------------


Algorithm \ref{alg:pcg} indicates that the computation involves four vector scaling operations in
steps \ref{pcg_scale1},  \ref{pcg_scale2} ,\ref{pcg_scale3} and  \ref{pcg_scale4},
two vector-vector multiplication operations of the inner products in steps \ref{pcg_dot1} and \ref{pcg_dot2},
and one matrix-vector multiplication operation in step \ref{pcg_mat}. 
$\mathcal{T}_c= (4 n^2 +2n^2+ 9n^2)\theta + \mathcal{T}_{p}  =15\frac{\mathcal{N}^2}{p}\theta+\mathcal{T}_{p}$,
where $\theta$ is the time unit per floating-point operation and
$\mathcal{T}_{p}$ is the execution time of preconditioning which may vary depending on the applied preconditioner.
For example, $\mathcal{T}_{p} =  \frac{\mathcal{N}^2}{p}\theta$ when a diagonal preconditioner is used. 
When the number of processes increases, $\mathcal{T}_c$ decreases and has a lower limit of zero.

Boundary updating occurs in the halo regions for each process, after operations like matrix-vector multiplication and non-diagonal preconditioning which requires one or more boundary layers. 
It is worth mentioning that only one boundary updating is needed each iteration even when a non-diagonal preconditioner is used, because every process keeps its own block and two extra halo layers in POP. 
The actual time depends on the network delay and the volume of the halo regions.
Since the halo size is $2$, the volume in each boundary is $2n$ and decreases as the number of processes increases. 
%It is worth mentioning that the preconditioning process usually requires at least one boundary layer,
%except for the diagonal preconditioner, which means that values on the outmost points are not updated after preconditioning. 
The total updating time for each iteration is then $\mathcal{T}_b =4\alpha +(4\times 2n)\beta=4\alpha +(\frac{8\mathcal{N}}{\sqrt{p}})\beta $,
where $\alpha$ is point-to-point communication latency per message and $\beta$ accounts for transfer time per byte (inverse of bandwidth). 
The updating time also decreases as the number of processes increases but has a lower bound of $4\alpha$.


There is only one global reduction in a ChronGear iteration,
thus the reduction time satisfies $\mathcal{T}_g= \log p \alpha$, assuming that a binomial tree approach is used in the reduction operation.
Obviously, $\mathcal{T}_g$ increases monotonically with the number of processes $p$.
Note that the global reduction has virtually no data exchange since there is only one number from each process.
%Let $T_0$ be the time unit of one floating-point operation and $B$ be the number of floating-point numbers transmitted by the network per second from process to process.
%Provided that the processor frequency and network bandwidth are $S_{cpu}$ and $B_{net}$, and that their efficiencies are $R_{cpu}$ and $R_{net}$, then $T_0 = R_{cpu} S_{cpu}^{-1}$, and $B = \frac{1}{8}R_{net}B_{net}$.
By considering the all three parts, the execution time of one diagonal preconditioned ChronGear solver step can be expressed as:
\begin{eqnarray}
%\begin{tabular}{l}
\label{t_pcg}
&\mathcal{T}_{cg}=\mathcal{K}_{cg} (\mathcal{T}_c + \mathcal{T}_b+\mathcal{T}_g )\nonumber \\
&=\mathcal{K}_{cg} [16 \frac{\mathcal{N}^2}{p}\theta + \frac{8\mathcal{N}}{\sqrt{p}}\beta +(4+\log p)\alpha]
%\end{tabular}
\end{eqnarray}
where $K_{cg}$ is the number of iterations in one ChronGear step which does not change with the number of processes \cite{hu2013scalable}.
Equation (\ref{t_pcg}) shows that the time required for computation and boundary updating decreases as the number of processes increases.
But the time required for global reduction increases with increasing
numbers of processes. Therefore, we expect the execution time of the
ChronGear solver to increase when the number of processors exceeds a certain threshold.
Indeed, the scaling behavior of ChronGear in the 0.1\degree POP is consistent with the above analysis (Fig.\ref{fig:ChronGearCOMP}).
The execution time of global reduction becomes dominant and increases when more than 2,634 cores are used.


\begin{figure}[!t]
\vspace{-10pt}
\begin{center}
	\includegraphics[width=1.0\linewidth]{ChronGear_comp.eps}
\end{center}
\caption[] {Time components of ChronGear Solver in 0.1\degree POP}
\label{fig:ChronGearCOMP}
\end{figure}
%However,it still inherits the poor scalability from PCG.  
%We tested the diagonal preconditioned ChronGear solver in  0.1\degree POP, and found that the scaling behavior is consistent with the above analysis. As shown in Fig.\ref{fig:ChronGearCOMP}, the execution time of global reduction becomes dominates in the ChronGear solver and increases when more than 2634 cores are used.

